\documentclass[review]{elsarticle}

\usepackage{lineno,hyperref}
\usepackage{amsmath}
\usepackage{booktabs}
\usepackage{color}
\usepackage[usenames,dvipsnames]{xcolor}
\usepackage{float}

\modulolinenumbers[1]

\journal{Remote Sensing of Environment}

%%%%%%%%%%%%%%%%%%%%%%%
%% Elsevier bibliography styles
%%%%%%%%%%%%%%%%%%%%%%%
%% To change the style, put a % in front of the second line of the current style and
%% remove the % from the second line of the style you would like to use.
%%%%%%%%%%%%%%%%%%%%%%%

%% Numbered
%\bibliographystyle{model1-num-names}

%% Numbered without titles
%\bibliographystyle{model1a-num-names}

%% Harvard
%\bibliographystyle{model2-names.bst}\biboptions{authoryear}

%% Vancouver numbered
%\usepackage{numcompress}\bibliographystyle{model3-num-names}

%% Vancouver name/year
\usepackage{numcompress}\bibliographystyle{model4-names}\biboptions{authoryear}

%% APA style
%\bibliographystyle{model5-names}\biboptions{authoryear}

%% AMA style
%\usepackage{numcompress}\bibliographystyle{model6-num-names}

%% `Elsevier LaTeX' style
\bibliographystyle{elsarticle-num}
%%%%%%%%%%%%%%%%%%%%%%%
\usepackage{setspace}
%\doublespacing
\singlespacing
\begin{document}

\newcommand{\edit}[1]{\textcolor{blue}{#1}}

\begin{frontmatter}

\title{Forecasting vegetation condition for drought early warning systems in pastoral communities in Kenya}


%% Group authors per affiliation:
%\author{Elsevier\fnref{myfootnote}}
\author{Adam B. Barrett$^{a,b}$, Steven Duivenvoorden$^{a,c}$, Edward Salakpi$^a$, James M. Muthoka$^{d}$, 
 John Mwangi$^{e}$, Seb Oliver$^{a,c}$ and Pedram Rowhani$^{d}$*}
\address{
	$^{a}$ \quad The Data Intensive Science Centre, Department of Physics and Astronomy, University of Sussex, Brighton BN1 9QH, UK\\
	$^{b}$ \quad Sackler Centre for Consciousness Science, Department of Informatics, University of Sussex, Brighton BN1 9QJ, UK \\
	$^{c}$ \quad Astronomy Centre, Department of Physics and Astronomy, University of Sussex, Brighton BN1 9QH, UK\\
	$^{d}$ \quad School of Global Studies, Department of Geography, University of Sussex, Brighton, BN1 9QJ, UK \\
	$^{e}$ \quad The National Drought Management Authority (NDMA), Lonrho House, Nairobi, Kenya \\
	*Corresponding author: P.Rowhani@sussex.ac.uk
	}




%% or include affiliations in footnotes:
%\author[mymainaddress,mysecondaryaddress]{Elsevier Inc}
%\ead[url]{www.elsevier.com}

%\author[mysecondaryaddress]{Global Customer Service\corref{mycorrespondingauthor}}
%\cortext[mycorrespondingauthor]{Corresponding author}
%\ead{support@elsevier.com}

%\address[mymainaddress]{1600 John F Kennedy Boulevard, Philadelphia}
%\address[mysecondaryaddress]{360 Park Avenue South, New York}


\begin{abstract}

\noindent Droughts are a recurring hazard in sub-Saharan Africa, that can wreak huge socioeconomic costs. Acting early based on alerts provided by early warning systems (EWS) can potentially provide substantial mitigation, reducing the financial and human cost. However, existing EWS tend only to monitor current, rather than forecast future, environmental and socioeconomic indicators of drought, and hence are not always sufficiently timely to be effective in practice. Here we present a novel method for forecasting satellite-based indicators of vegetation condition. Specifically, we focused on the Vegetation Condition Index (VCI) over pastoral livelihood zones in Kenya, which is the indicator used by the Kenyan National Drought Management Authority (NDMA). Using data from MODIS and Landsat, we apply linear autoregression and Gaussian processes modeling methods and demonstrate highly skillful forecasting several weeks ahead. As a bench mark we predicted the drought alert marker used by NDMA (3 month VCI$<35$). Both of our models were able to predict this alert marker four weeks ahead with a hit rate of around 89\% and a false alarm rate of around 4\%, or 81\% and 6\% respectively six weeks ahead. The methods developed here can thus identify a deteriorating vegetation condition well and sufficiently in advance to help disaster risk managers act early to support vulnerable communities and limit the impact of a drought hazard.
\end{abstract}

\begin{keyword}
Drought; Forecasting; Early Warning Systems; Disaster Risk Reduction; Landsat; MODIS
\end{keyword}

\end{frontmatter}

\linenumbers

\section{Introduction}

Droughts are a major threat globally as they can cause substantial damage to society, especially in regions that depend on rain-fed agriculture. They particularly impact food security by significantly reducing agricultural production \citep{Lesk} and raising food prices \citep{Nelson3274,BROWN201531}, which often leads to increased levels of malnutrition, migration, disease, and other health concerns \citep{10.1093/rsq/hdr006,Stanke}. %Since 2000, there have been 319 drought events reported \citep{emdat}, which together have killed over 21,000 people and affected almost 1.4 billion others. 
The majority of droughts occur in sub-Saharan Africa \citep{emdat} where many communities rely on predictable rainfall patterns for their livelihood. 


In East Africa, the main economic activity in the arid and semi-arid lands (ASAL) is subsistence rain-fed agriculture, as well as livestock farming using pastures and grasslands as the main source of fodder. As a result, the pastoral and agro-pastoral communities who live in these drylands are particularly vulnerable to drought 
%have dealt with rainfall variability and drought over centuries by developing extensive adaptation and mitigation strategies to reduce their vulnerability to these shocks
\citep{Nyong2007,orindi2007pastoral}, especially since their existing coping strategies have been compromised by population growth and land use change in recent years \citep{Galvin2001ImpactsOC}. %Additionally, while there is some uncertainty in the climate models \citep[IPCC,][]{stocker2013climate}, rainfall variability is expected to increase in the region \citep{Tierneye1500682,yang2018brief}. These factors in combination will make it harder for indigenous knowledge systems to deal with droughts, and exacerbate the problems created by droughts.  
Governments and donor agencies in the region have thus developed several tools and early warning systems (EWS) to mitigate the impact of droughts on pastoralists.

Most EWS tend to monitor current key biophysical and socio-economic factors to assess the possible exposure of vulnerable people to specific hazards. However, once the impacts are visible, it may be too late to mitigate the consequences \citep{kogan}. As a consequence, there is growing interest in moving toward a proactive humanitarian approach to disasters by developing preparedness actions based on climate forecasts \citep{nhess-15-895-2015,lopez2018bridging, wilkinson2018forecasting}. Additionally, it is estimated that being better prepared before a drought hits significantly reduces the costs and losses from these disasters \citep{venton2012economics}. Hence, EWS now increasingly include expert knowledge and qualitative assessments of seasonal climate forecasts to assess the future development of food security, and define actions to mitigate possible losses \citep{nhess-15-895-2015,TozierdelaPoterie2015}. However for drought conditions, a meteorological drought does not always lead to negative agricultural outputs \citep{BHUIYAN2006289}. There is thus a growing interest to include forecasts of the impacts of these hazards\citep{wmo2015wmo,nhess-2018-26,Sutanto2019}.

%Within East Africa, the Famine Early Warning Systems Network (FEWS NET) monitors food security through data collection and a deep understanding of the livelihood patterns in the region. A team of experts and analysts will also look at seasonal climate forecast to estimate future food security outcomes using scenario development \citep{FEWS}. 
In Kenya, following several periods of intense drought, the government established the National Drought Management Authority (NDMA) in 2016, to set up and operate a drought EWS, as well as to establish drought preparedness strategies and contingency plans. The NDMA provides monthly bulletins assessing food security in the 23 ASAL regions using current  biophysical (e.g., rainfall, vegetation condition) and socio-economic (production, access, and utilisation) factors. One key biophysical indicator used by the NDMA drought phase classification is based on the Vegetation Condition Index  \citep[VCI,][]{KOGAN199591,rs8040267,RULINDA201132,ROJAS2011343}. 

The VCI, which expresses the Normalized Difference Vegetation Index (NDVI) in terms of where it currently lies within its expected range for the given pixel, is one of a number of satellite-based indicators that have been developed to detect and monitor drought \citep{zargar2011review}. While there is little agreement between VCI and precipitation-based meteorological drought indicators \citep{quiring2010evaluating,BHUIYAN2006289}, it is strongly linked to agricultural production and widely used to identify drought onset, intensity, duration, and impact \citep{rs8030224}. The NDMA uses the 3-month averaged VCI (VCI3M) in its operational EWS \citep{rs8040267}. Once the VCI3M goes below a threshold of 35, the NDMA triggers a rapid food security assessment and has access to the National Drought Contingency Fund in order to implement its preparedness strategies and contingency plans.


%Based on these factors, the bulletins include a qualitative evaluation of food security outcomes in the months ahead. {\color{red} Shouldn't the next sentence be in the discussion, not here, because we are not addressing consequences in this paper and also our premise is that the current EWS do not forecast at all [Seb]}{\color{green} Agreed! we should have whole paragraph on using forecasts and moving from hydro to impact-based [Pedram]}. However, EWS should move from forecasting hydro-meteorological events toward estimating the expected consequences of hazards, i.e. impact-based forecasting, to identify more effective early action protocols \citep{wmo2015wmo,nhess-2018-26}.

%{\color{green} the next paragraph does not read well, maybe should be omitted and moved to the discussion where we bring in other 'forecast' papers? Pedram}
%The NDVI and VCI provide timely and regular assessment of vegetation health over large spatial areas which is useful to pastoralists  since they strongly rely on forage availability to keep their livestock. \edit{Forecasts of these indicators within EWS} would allow local and national stakeholders \edit{be better prepared and act early to} support these communities in times of drought. Recently, several studies have highlighted the potential of these satellite-based Earth observation data to forecast agricultural productivity \citep{ZAMBRANO201815} and seasonal forage availability \citep{VRIELING201644}.

The main goal of this paper is to explore machine-learning techniques to forecast (up to six weeks ahead) the vegetation indices that are commonly used in the pastoral areas of Kenya to monitor droughts. Based on NDMA's experience, we particularly focused on the pastoral livelihood zones as the VCI3M is more reliable in identifying drought condition for grazing and browsing in the more arid regions of the country.  Several studies have developed statistical and machine-learning approaches \citep{udelhoven,meroni2014early,Zambrano:2018,vrieling2016early} to predict end-of-season crop, forage and biomass production. Recently, \cite{matere2019predictive} developed a decision support tool based on a mechanistic model to estimate 6-monthly forecasts of forage condition. Here, we specifically focus on Gaussian Process modelling \cite[GP,][]{gpm}, and linear autoregressive (AR) modelling \citep[e.g.][]{Hamilton94} to forecast NDVI and the 3-month VCI (VCI3M), which are derived from both Landsat (every 16 days at 30\,m resolution) and the MODerate resolution Imaging Spectroradiometer (MODIS - daily data at 500\,m resolution). GP modelling uses kernel-based non-parametric Bayesian inference on the structure of correlations between observations, and is widely applied to classification, interpolation, change detection and forecasting problems \citep{955315,Chandola2010SCALABLETS,7487896,rs11050481}. %For an overview on the principles of GPs, and how they have previously been applied throughout remote sensing, see \cite{7487896}. 
Linear AR is the regression of future observations on past observations, assuming a linear dependence. Previously it has been performed on monthly (i.e.~temporally more sparse) NDVI data, see for example \cite{Asoka:2015} and \cite{Papagiannopoulou}, with mixed results in terms of forecasting potential ($R^2$-scores between 0 and 0.4 at a lead time of one month). 



%\edit{Their drought early warning system includes clear thresholds of VCI3M which are used to classify drought severity in the ASAL regions of Kenya [Klisch et al]. Here, we extract the indices from data derived from the Landsat mission (every 16 days at 30\,m resolution) and the MODerate resolution Imaging Spectroradiometer (MODIS - daily data at 500\,m resolution).}

%\edit{Machine-learning techniques offer a data-driven, empirical route to these forecasts. Many different data inputs could be used to forecast these vegetation indices (e.g. precipitation and precipitation forecasts. However, perhaps the most simple is to use the past history of the indices themselves. This has the practical benefits of readily available data over large areas. It is also likely to work as these indices are subject to plant growth and climate cycles giving periodic behaviour on large temporal scales that can be empirically modelled while external perturbations, such as water availability, have persistent impact providing correlations on short temporal scales. The existing EWS itself implies that the indices have forecasting power as moderately low indexes are labelled ``alert" implying they might precede lower ``alarm" levels. }

%{\color{red} something about the Boku paper here}

%\edit{The machine-learning techniques we attempt are Gaussian Process regression \cite[GP,][]{gpm}, and linear autoregressive (AR) modelling \citep[e.g.][]{Hamilton94}.} GP regression uses kernel-based non-parametric Bayesian inference on the structure of correlations between observations, and is widely applied to classification, interpolation, change detection and forecasting problems \citep{955315,Chandola2010SCALABLETS,7487896,rs11050481}. For an overview on the principles of GPs, and how they have previously been applied throughout remote sensing, see \cite{7487896}. Linear AR is the regression of future observations on past observations, assuming a linear dependence. This has previously been performed on monthly (i.e.~temporally more sparse) NDVI data, see for example \cite{Asoka:2015} and \cite{Papagiannopoulou}, with mixed results in terms of forecasting potential ($R^2$-scores between 0 and 0.4 at a lead time of one month).  


%\edit{This paper details the data we have worked with, including the locations chosen and specific preprocessing applied. We then explain our forecasting methods in detail. We present results and analysis of accuracy or skill of our methods through hindcasting. Finally we discuss the pros and cons of our methods and consider the route to implementation in real early warning systems.}

\section{Study area}
In Kenya, the livestock sector accounts for 13\% of the national GDP and 43\% of its agricultural GDP. Livestock farming mainly occurs in the ASAL which cover about 80\% of the country \citep{UNDP2013,FAO2014}. In these regions, the pastoral communities rely on pastures and grasslands as the main source of fodder \citep{Behnke2011}. Thus, providing information on pasture productivity to these communities is key in times of drought.


\begin{figure}
	\centering
	\includegraphics[trim = 80mm 10mm 30mm 10mm,width=5.0 cm]{figures/MyMap6A.pdf} \qquad \includegraphics[trim = 20mm 0mm 0mm 40mm,width=5.0 cm]{figures/MyMap6B.pdf}
	\caption{Maps of Kenya showing (a) the livelihood zones from which pixels were sampled for analysis, and (b) land-cover classification (according to the MODIS MCD12Q1 data). Analyses were performed for 29 regions, defined by pastoral livelihood zone and county intersections.} \label{fig:l_zone}
\end{figure}

% \begin{figure}[!h]
% 	\centering
% 	\includegraphics[trim = 45mm 0mm 0mm 0mm,width=5.0 cm]{figures/MyMap6.pdf} 
% 	\caption{Maps of Kenya showing (a) the livelihood zones from which pixels were sampled for analysis, and (b) land-cover classification (according to the MODIS MCD12Q1 data). Analyses were performed for 29 regions, defined by pastoral livelihood zone and county intersections.} \label{fig:l_zone}
% \end{figure}


%Following several periods of intense drought, the government in Kenya established the NDMA in 2016, to set up and operate a drought early warning system (DEWS), as well as to establish drought preparedness strategies and contingency plans (GoK){\color{green} this should be reference - I will have to dig it out. And remove the previous acronym DEWS as it is not used much}. 

%One key biophysical indicator used by the NDMA drought phase classification is the VCI \citep{rs8040267}. {\color{blue} The VCI, which expresses the NDVI as a fraction of its limiting range for a given pixel [Kogan et al 1995, (see Equation \ref{vci_eq})], is one of a number of satellite-based indicators that have been developed to detect and monitor drought [Zargar et al. 2011]. While there is little agreement between VCI and precipitation-based meteorological drought indicators [Quiring and Ganesh 2010, Bhuiyan et al 2006], it is strongly linked to agricultural production and widely used to identify drought onset, intensity, duration, and impact [Jiao et al 2016, Klisch and Atzberger]. The NDMA uses the 3-month averaged VCI (VCI3M)  in its operational EWS. Once the VCI3M goes below 35, the NDMA triggers a rapid food security assessment and has access to the National Drought Contingency Fund in order to implement its preparedness strategies and contingency plans [GoK 2013].}


For the ASAL regions, the NDMA reports every month the VCI3M value at county level as well as over the different livelihood zones within the county. This study focused on the 10 (agro)-pastoral livelihood zones (see Fig \ref{fig:l_zone}), which cross 15 counties. The names of the 29 livelihood zone county intersections can be found in \ref{app}. 
%FEWS NET shape files were used to define these regions, and to demarcate the Landsat and MODIS pixels from which to sample data.




% \section{Data preprocessing} \label{sec:data}

% \edit{This research is based on two satellite-based Earth observation datasets, Landsat and MODIS.} A comparison between them, and justification of data selection can be found in \ref{sec:datasets}.  \edit{It should be noted that the analysis is based on a random subsample of the pixels within each of the 29 pastoral livelihood zone and county intersections of interest (Fig. \ref{fig:l_zone}).}

% A summary of the entire work from data preparation to forecasting drought can be seen in Figure \ref{fig:flowchart}.

% \subsection{Landsat}
% Landsat-5, 7 and 8 \citep{royetal} red and near infrared (NIR) surface reflectances and quality assessment (QA) data over the 10 pastoral livelihood zones of Kenya, from \edit{January 1st, 2000 to February 1st, 2019}, were obtained using the United States Geological Survey (USGS) EarthExplorer. Specifically, data from 1\,000 pixels within each region were drawn from the Level-1 Precision Terrain (L1TP) processed dataset, which has well-characterized radiometry and is inter-calibrated across the different Landsat sensors. The spatial resolution of these data is 30m and the repeat interval is 16 days. Using the QA data, observations classified as clear were kept. \edit{Pixels with fewer than half of the observations over the full time period were discarded (and replaced with an alternative random selection, with a few exceptions, see Table \ref{tab:NDVI_LS}}). %Landsat-5 data were available up until November 2011 (albeit with several large gaps), Landsat-7 data were available for the whole time-period, and Landsat-8 data were available from March 2013. 
% \edit{The surface reflectances} were combined to obtain 
% NDVI. %\edit{\footnote{${\rm NDVI}=({\rm NIR}-{\rm Red})/({\rm NIR}+{\rm Red})$}} 
% % from which we derived the NDVI anomaly and VCI {\color{blue}(see Eq. \ref{vci_eq})}. 
% %, \edit{out of those for which at least half of the observations were present and labelled good quality.}\footnote{\edit{Except for a few regions, for which this threshold had to be dropped in order to be able to obtain 1\,000 pixels, see Table \ref{tab:NDVI_LS} in the Appendix.}} 

% \edit{The temporal aggregation and gap-filling on the Landsat data was done using Gaussian Process (GP) modelling. For a given pixel, the GP modelling took raw data as input, fit a temporal correlation structure to the data, and used this to output a time series of expected NDVI values, with observations provided every Saturday over the studied time period; see \ref{sec:GPexplain} for details. Two versions of GP gap-filling were carried out, which we refer to as forecast mode and non-forecast mode. For the non-forecast mode, the full time series from the given pixel were used to train the GP. The non-forecasting mode was used as the ``ground truth'' to test forecasts against. The forecast mode, by contrast, only used data up to a certain date, whichever date a forecast was being attempted from - since when doing forecasting with a near real-time data stream, one does not have access to future data. }
% % We note that the red and NIR bands are defined by slightly different spectral ranges for the different Landsat missions, and that this can lead to slightly different NDVI values for the same vegetation. More specifically, Landsat-8 measures slightly higher values for the NDVI of dry shrub-lands than Landsat-7, while at higher values of NDVI there is more convergence \citep{LS}. The data from all three Landsat missions where used to train the GP's (see Section \ref{sec:Method}).

% %1\,000 random pixels were selected for processing.

% \subsection{MODIS}
% \edit{The NDVI was also derived from the surface reflectances obtained from the daily, 500-meter resolution} MODIS Terra/Aqua Nadir BRDF-Adjusted Reflectance product \citep[MCD43A4,v006;][]{Schaaf2015}. Data from February 22nd, 2000 up to February 1st, 2019 were acquired via the NASA Land Processes Distributed Active Archive Center. QA maps files with binary quality flags were used to remove poor quality data resulting from cloud or unreliable BRDF corrections. \edit{We manually selected 100  pixels (distributed thoughout each region) that had been  identified  as grassland} by the MODIS land cover classification maps (MCD12Q1,v006). 
% % using Application for Extracting and Exploring Analysis Ready Samples (AppEEARS),
%  %The spatial resolution of \edit{this product} is 500m and the repeat interval is 1 day, \edit{with the daily data being 16 day composites of raw observations at 1 to 2 day intervals}. 


% \edit{Weekly NDVI composites were obtained by taking the mean of all available data over a 7-day time period. Gaps in the weekly time series were then filled using quadratic interpolation. Gaps longer than 6 weeks were \edit{left unfilled}, see \ref{sec:Lmax} for  details on this. The gap-filled time series were then smoothed using the Savitzky-Golay method \citep{Savitzky1964} to filter high-frequency measurement noise. The smoothing involved fitting, for each pixel, a polynomial to a window centred on the observation, and then replacing that observation with the output of the polynomial fit. Following \cite{Chen2004} and  \cite{bg-10-4055-2013}, the polynomial order was set to 2 (i.e.~quadratic function) and the length of the window was set to 7 weeks. (Note that the combined interpolation and smoothing procedure does 2 rounds of quadratic interpolation where there are gaps, but that these are distinct: the interpolation fills a gap of up to 6 weeks with one quadratic function, while the smoothing modifies only one observation per fitted quadratic function.)}




\newpage

\begin{figure} 
	\centering
	\includegraphics[trim = 0mm 0mm 0mm 20mm,width=11.5 cm]{figures/Flowchart-V2.pdf}
	\caption{A flow chart of the data processing and analysis.} \label{fig:flowchart}
\end{figure}
\newpage


\section{Methods} \label{sec:methods}
This research is based on two satellite-based Earth observation datasets, Landsat and MODIS. Description and justification of data selection, and a comparison between the two datasets can be found in \ref{sec:datasets}. It should be noted that the analysis is based on a random subsample of the pixels within each of the 29 pastoral livelihood zone and county intersections of interest (Fig.~\ref{fig:l_zone}). A summary of the entire work from data preparation to forecasting drought can be seen in Fig.~\ref{fig:flowchart}.

% A detailed description of the various preprocessing steps}, a comparison between the two datasets, and justification of data selection can be found in \ref{sec:datasets}.  

\subsection{Data preprocessing} \label{sec:preprocessing}

\subsubsection{Landsat}
\linelabel{review:GP_sketchL}
\begin{figure}
\label{review:GP_sketch}
    \centering
    \includegraphics[width=10cm]{figures/Barrett2019sketch2-GP.png}
    \caption{Illustration of the GP approach used for the Landsat data. In ``forecast mode", the correlations in the data up to a given date (black) furnish a GP model (green), which can then be used for forecasting. In the ``non-forecast mode", the entire time series is used to train the GP, and provide a ground truth for the forecast.}
    \label{fig:GP_illustration}
\end{figure}

Temporal gridding and gap-filling on the Landsat data was done using Gaussian Process (GP) regression. For a given pixel, the GP regression took raw data as input, fit a temporal correlation structure to the data, and used this to output a time series of expected NDVI values, with observations provided every Saturday over the studied time period; see Figure~\ref{fig:GP_illustration} for an illustration and \ref{sec:GPexplain} for details. Two versions of GP gap-filling were carried out, which we refer to as forecast mode and non-forecast mode. For the non-forecast mode, the full time series from the given pixel were used to train the GP. The non-forecasting mode was used as the ``ground truth'' to test forecasts against. The forecast mode, by contrast, only used data up to a certain date, whichever date a forecast was being attempted from - since when doing forecasting with a near real-time data stream, one does not have access to future data. 
% We note that the red and NIR bands are defined by slightly different spectral ranges for the different Landsat missions, and that this can lead to slightly different NDVI values for the same vegetation. More specifically, Landsat-8 measures slightly higher values for the NDVI of dry shrub-lands than Landsat-7, while at higher values of NDVI there is more convergence \citep{LS}. The data from all three Landsat missions where used to train the GP's (see Section \ref{sec:Method}).

%1\,000 random pixels were selected for processing.

\subsubsection{MODIS}

Weekly NDVI composites were obtained for each pixel by taking the mean of all available data over a 7-day time period. Gaps in the weekly time series were then filled using quadratic interpolation. Gaps longer than 6 weeks were left unfilled, see \ref{sec:Lmax} for  details. The gap-filled time series were then smoothed using the Savitzky-Golay method \citep{Savitzky1964} to filter high-frequency measurement noise. The smoothing involved fitting, for each pixel, a polynomial to a window centred on the observation, and then replacing that observation with the output of the polynomial fit. Following \cite{Chen2004} and  \cite{bg-10-4055-2013}, the polynomial order was set to 2 (i.e.~quadratic function) and the window length to 7 weeks. (Note that the combined interpolation and smoothing procedure does two rounds of quadratic interpolation where there are gaps, but that these are distinct: the interpolation fills a gap of up to 6 weeks with one quadratic function, while the smoothing modifies only one observation per fitted quadratic function.)



\subsection{Indices}
For both datasets, the weekly NDVI time series from the sampled pixels were converted to NDVI anomalies (i.e., the seasonal mean-subtracted NDVI, sometimes referred to as absolute anomaly) and VCI, which at time point $i$ is defined as:

\begin{equation}
\rm{VCI}_i = 100 \times \frac{\rm{NDVI}_i - \rm{NDVI}_{min,i}}{ \rm{NDVI}_{max,i} - \rm{NDVI}_{min,i}},
\label{vci_eq}
\end{equation}
%
where $\rm{NDVI}_{min,i}$ and $\rm{NDVI}_{max,i}$ are the minimum and maximum observed values for the NDVI of the pixel for the week of the year at time point $i$. The data within each livelihood zone and county intersection were aggregated taking the mean of the sampled pixels at each time point. Thus forecasting was applied on a single time series for each region (see Fig. \ref{fig:ndvi_lk} for an example time series). Finally, VCI3M was calculated as the mean VCI across the 12 weeks leading up to the given time point. 

With Landsat data, the mean, maximum and minimum value for the NDVI in \eqref{vci_eq} was computed using the non-forecast mode GP interpolated time series. Then forecast mode and non-forecast mode versions of each index were created. With the MODIS data, since large gaps were unfilled, whenever there were fewer than 25 individual pixel observations from a particular region at a given time, it was decided that there should be no datum in the aggregate NDVI anomaly time series (i.e., there should be a gap in the time series). Additionally, if the current aggregate NDVI observation was not present, a gap was placed in the VCI3M time series. Else, the mean was taken over all present observations from the most recent 12 weeks.

\begin{figure}
	\centering
	\includegraphics[trim = 50mm 0mm 0mm 0mm,width=13.5 cm]{figures/NDVI2.pdf} 
	\caption{Aggregate NDVI time series from the intersection of Baringo county and livelihood zone 24. The top panel shows MODIS data, and the bottom panel shows Landsat data (solid lines), processed as described in the text. The dotted lines in each panel show forecasts at a lead time of 2 weeks, from the AR method on the MODIS data, and from the GP method on the Landsat data.} \label{fig:ndvi_lk}
\end{figure}

\subsection{Forecasting}\label{forecast_method}
Machine-learning techniques offer a data-driven, empirical route to forecasting. Many different data inputs could be used to forecast these vegetation indices (e.g. precipitation and precipitation forecasts). However, perhaps the most simple is to use the past history of the indices themselves. This has the practical benefits of readily available data over large areas. It is also likely to work as these indices are subject to plant growth and climate cycles giving periodic behaviour on large temporal scales that can be empirically modelled while external perturbations, such as water availability, have persistent impact providing correlations on short temporal scales. Forecasts of NDVI anomaly and VCI3M were calculated using two separate methods, respectively based on Gaussian Process modelling (GP) and linear autoregressive (AR) modelling.

GP forecasting was performed by fitting a GP to the forecast mode aggregate time series for the index in question, and then using the GP to extrapolate. For details on GP modelling, see \ref{sec:GPexplain}. The key step involved fitting a temporal correlation structure to the data, i.e.~a kernel $k(t,t^{\prime})$ that describes the covariance between the index at any two times $t$ and $t^{\prime}$. The kernel with the highest evidence was the Radial Basis Function (RBF):

\begin{align}
	k_{\rm{RBF}}(t,t') &= \sigma_{\rm{RBF}}^2 \exp{\Big( -0.5 \frac{|t-t'|^2}{l_{\rm{RBF}}^2} \Big) }\,,
\end{align}
and the modelling was carried out with the best fit version of this.

AR forecasting was performed with the following model-fitting and extrapolation method. For forecasting $n$ weeks ahead, the following model was fit:

\begin{equation}
X_{t+n}=\sum_{i=0}^{p-1}a_iX_{t-i}+\epsilon_t\,, \label{eq:AR1}
\end{equation}
where $X$ is the index in question, subscripts denote the date (week), $a_i$ are model coefficients, $\epsilon_t$ are the residuals (i.e.~the errors), and $p$ is called the model order. (This model assumes zero mean, so for VCI3M, the mean was removed prior to fitting the model, and then added back again after using the model to forecast the deviation from the mean.) Fitting the model to a segment of data involved finding the model coefficients that gave the minimum sum-square error, i.e.~led to residuals with the minimum variance. To make a forecast, the model was fit using the most recent $T$ consecutive observations (where $T$ was a parameter choice to be determined), and then used to predict the observation $n$ weeks after the most recent observation. This forecasting method was carried out along the entire available time series, fitting a distinct model to each segment of length $T$. A search for optimal model orders and training segment lengths found that forecast quality, as measured by root mean square error (RMSE), plateaued at $T=200$ and $p=3$.%\footnote{High values of $p$ up to 20 were also explored with LASSO regression (a regularisation procedure), but this led to inferior forecasting.}



% To check that results were not strongly dependent on the choice of maximum allowed interpolation length $L_{\mathrm{max}}$ in the gap filling (see Secion \ref{sec:gapMODIS}) the method was applied to the MODIS data processed with $L_{\mathrm{max}}=4$ and 8, in addition to $L_{\mathrm{max}}=6$. As can be seen in Table \ref{tab:IL choices}, results did not substantially depend on the choice of this parameter. 

% For comparison, this method was also applied to the Landsat data. AR models were fit to the forecast mode time series, and predictions from the model were then compared to corresponding data on the non-forecast mode time series. 

\subsection{Forecast assessment}
Several metrics were used to assess the performance of the forecast methods tested on the data. In addition to RMSE, the $R^2$-score and the percentage of standard deviation remaining, $S$, were used. These are given by:

\begin{align}
	R^2{\text -}{\rm score} &= 1 - \frac{\sum_i (y_i - f_i)^2 }{\sum_i (y_i - \bar{y})^2}, \label{eq:r2} \\
	S &= 100 \times \frac{\sqrt{ \sum_i (y_i-f_i)^2 }}{\sqrt{  \sum_i (y_i-\bar{y})^2 }} \label{eq:S}\,,
\end{align}
where the $y_i$ are the true data, and the $f_i$ are the forecasts. Note that $S\equiv 100\times\sqrt{1-R^2{\text -}\rm{score}}$. To test for bias, linear regression of actual index on forecast index was performed, and slope and intercept computed. Finally, receiver operating characteristic (ROC) curves were constructed for forecast-based drought-alert detection. 

%{\color{red} SEB: can you please also check the wording of this section as I brought in some text from the results - Pedram}
These performance indicators were also used to assess the sensitivity of our methods in space (comparing the results by region) and in time (to account for seasonality). Additionally, the forecast methods were evaluated for various drought categories \citep{rs8040267}, compared against a persistence forecast, and the impact of data gaps on forecast performance was analysed.

Forecasts on the MODIS data were assessed from January 1st 2014 onward, which was approximately the earliest date for which there were sufficient prior data for the AR method to be applied. Forecasts on the Landsat data were assessed only from January 1st 2014 onward, since the GP gap-filling method required training on data up to this date.

\section{Results} \label{sec:results}

\subsection{Forecast value accuracy}

\begin{figure}
	\centering
	\includegraphics[trim = 30mm 0mm 0mm 10mm,width=5.4 cm]{figures/contour_LsatNDVI_Anom2.pdf} \qquad \includegraphics[trim = 21mm 0mm 0mm 0mm,width=5.4 cm]{figures/contour_adam_NDVI1.pdf}
	\caption{Contour plots of NDVI anomaly against two, four and six weeks NDVI anomaly forecasts. (a,c,e) show forecast performance for the GP method on Landsat data, and (b,d,f) show  forecast performance for the AR method on MODIS data (across the 19 regions for which a 4 week forecast was possible more than 50\% of the time, see main text for details).} \label{fig:contourNDVI}
\end{figure}

\begin{figure}
	\centering
	\includegraphics[trim = 30mm 35mm 0mm 10mm,width=5.4 cm]{figures/contour_LsatVCI3M1.pdf} \qquad \includegraphics[trim = 30mm 35mm 0mm 10mm,width=5.4 cm]{figures/contour_adam1.pdf}
	\caption{Contour plots of VCI3M against two, four and six weeks VCI3M forecasts. (a,c,e) show forecast performance for the GP method on Landsat data, and (b,d,f) show forecast performance for the AR method on MODIS data (across the 19 regions for which a 4 week forecast was possible more than 50\% of the time, see main text for details).} \label{fig:gpcontour}
\end{figure}

%VCI3M
% ['r2:0.99_rmse:1.941_slp:1.024_int:-0.622',
 % 'r2:0.946_rmse:4.41_slp:1.059_int:-1.551',
 %'r2:0.854_rmse:7.286_slp:1.111_int:-2.955']
 %anom
% ['r2:0.662_rmse:0.031_slp1.137_int:0.006',
% 'r2:0.418_rmse:0.04_slp1.248_int:0.009',
% 'r2:0.208_rmse:0.047_slp1.378_int:0.011']
\begin{table}
	\small
	\caption{ Performance statistics of NDVI anomaly and VCI3M forecasts with lead times of 2, 4 and 6 weeks.} \label{tab:stats}
	\centering
	\begin{tabular}{l|ccc|ccc} 
		\toprule
		& & \textbf{Landsat GP} & & &\textbf{MODIS AR} \\
		%\textbf{forecast}
		& \textbf{2} & \textbf{4} & \textbf{6} & \textbf{2} & \textbf{4} & \textbf{6} \\
		%\textbf{ahead}
		& \multicolumn{3}{c|}{\textbf{weeks }}& \multicolumn{3}{c}{\textbf{weeks }}\\

		\midrule
		\textbf{NDVI }&&&&&\\
		\textbf{anomaly:}&&&&&\\
		$R^2$-score  &0.66  &0.42  &0.21  &0.85& 0.55& 0.33\\
		RMSE &0.031  &0.04  &0.047  &0.025& 0.043& 0.053\\
		slope &1.14  &1.25 &1.38 &0.99&0.99& 0.97\\
		intercept &0.006 &0.009 &0.011&-0.00& -0.00& -0.00\\
		\midrule
		\textbf{VCI3M:}&&&&&\\
		$R^2$-score  &0.99  &0.95  &0.85  &0.99& 0.96& 0.88\\
		RMSE &1.94  &4.41  &7.286  &1.8& 4.3& 7.0 \\
		slope &1.024  &1.059 &1.11 &1.00&1.00& 1.00\\
		intercept &-0.62  &-1.55 &-2.95&-0.0& -0.1& -0.0\\
		\bottomrule
	\end{tabular}
\end{table}

The GP and AR forecasting methods were applied, on each of the two datasets, to regional aggregate NDVI and VCI3M time series. We focus on performance results of GP forecasting on Landsat data and AR forecasting on MODIS data since these two combinations of data and forecasting method performed the best (as measured by $R^2$-score, see \ref{app}). Contour plots of forecast against actual data for two, four and six week forecasts are shown in Fig.~\ref{fig:contourNDVI} for NDVI anomaly, and Fig.~\ref{fig:gpcontour} for VCI3M. Table \ref{tab:stats} shows the $R^2$-scores, RMSE, slope and intercept from each of these plots, and demonstrates that there is substantial forecast skill from each method at each lead time ($R^2$-scores are substantial), and that the forecasts are unbiased (slopes are all approximately 1, and intercepts approximately 0). The much higher $R^2$-scores for VCI3M compared to NDVI anomaly is due to the fact that VCI3M is the 12 week composite of weekly VCI observations, and thus there is much greater correlation between VCI3M at time $t$ and VCI3M at time $t+(2,$ 4, 6) weeks due to overlap in the composited weeks. Notwithstanding this, each forecasting method performed substantially better than a persistence forecast of VCI3M. For example, the AR method on the MODIS data achieved an RMSE of approximately half that of the persistence forecast for a lead time of 4 weeks, see Fig.~\ref{fig:persistence} in the Supplementary Material.

For both methods, the forecast time series lag behind the true time series, since changes not foreseen by the models are incorporated only once they are observed, see Fig.~\ref{fig:ndvi_lk} for examples. The two methods tend to make different types of error. The GP method has a tendency to err in the direction of a reversion to the long term mean value. This is because \textit{a priori} to taking into account the most recent observations, the model assumes the forecast observation will be equal to the long term mean. By contrast, the AR method sometimes mistakenly predicts a continuation of the recent trend. This is because the model assumes a continuation of the recent frequency profile, so if a faster-than-average trend is seen in either direction, the trend will be predicted to continue \citep{Hamilton94}.

% Fig.~\ref{fig:timeline} shows that the same approximately linear relation between NDVI anomaly and lagged NDVI anomaly holds for the forecast and non-forecast mode time-series, and hence gives some validation of the method.\footnote{The assumption, under GP interpolation and extrapolation, that data anomalies follow a multivariate Gaussian distribution necessarily also assumes that the relationship between past and present anomalies are linear. This follows from the fact that a stationary Gaussian AR process must be linear \citep{Barrett:2010}.} Plots of NDVI anomaly against lagged NDVI anomaly showed that there is a strong linear relation between present and past observations of NDVI anomaly, see Fig.~\ref{fig:timeline}, thus it is anticipated that the linear AR model captures a good degree of the non-random relationship between past and future NDVI anomaly observations.

% \begin{figure}
% 	\centering
% 	\includegraphics[trim = 50mm 4mm 0mm 3mm,width = 14 cm]{figures/timeline.pdf} 
% 	\caption{The approximately linear relation between Landsat NDVI anomaly and NDVI anomaly lagging at two, four and six weeks in the past. Blue dots show the Landsat data, processed with non-forecast mode GPs, which was taken as `ground truth'. Red dots show the data processed with forecast mode GPs, from which forecasts were generated.} \label{fig:timeline}
% \end{figure}


Due to the presence of non-interpolated gaps in the MODIS time series, there were weeks when a forecast assessment was not carried out on these data, see \ref{sec:Lmax} for details.  For 15 of the regions, a 4 week forecast could be made on more than 90 percent of weeks; however, for some of the more cloudy/wet regions, a forecast could rarely be made, see Table \ref{tab:VCI3m_MODIS} in the Supplementary Material.

Additional checks were included to test the sensitivity of the methods to drought severity and seasonality. 
%First, we analysed whether our forecast skill depended on the true vegetation condition. 
The methods, when computed separately for each of the five categories on the NDMA drought scale \citep{rs8040267}, perform better when there was a state of drought than when the vegetation condition was normal (Table \ref{tab:RMScategories}).
Moreover, both methods do not show substantial seasonal differences, although RMSE was generally somewhat elevated for some of the January/February dry season (Fig.~\ref{fig:seasonal}).
%we plotted RMSE of VCI3M forecast against the week of the year (Fig.~\ref{fig:seasonal}) to check whether our forecast skill depended on the time of year. The seasonal differences in RMSE are not substantial on the scale on which VCI3M varies, although RMSE was generally somewhat elevated for some of the January/February dry season.}

%{\color{green} I've put this paragraph back into the results - Adam can you check its in the right place}




\begin{table}
	\small
	\caption{RMSE in VCI3M forecast, for the true vegetation condition belonging to the different categories of drought, at lead times of 2, 4 and 6 weeks. Drought categories are defined by the VCI3M index: wet by VCI3M$>$50;  normal by 35$<$VCI3M$<$50; moderate drought by 20$<$VCI3M$<$35; severe drought by 10$<$VCI3M$<$20; and extreme drought by VCI3M$<$10.
	(The extreme drought criterion was not met in any of the Landsat data.) } \label{tab:RMScategories}
	\centering
	\begin{tabular}{l|ccc|ccc} 
		\toprule
		\textbf{Drought category} & & \textbf{Landsat GP} & & &\textbf{MODIS AR} \\
		& \textbf{2 } & \textbf{4 } & \textbf{6 } & \textbf{2 } & \textbf{4 } & \textbf{6}\\
		& \multicolumn{3}{c|}{\textbf{weeks}} &  \multicolumn{3}{c}{\textbf{weeks}}\\
		\midrule
		Wet  & 2.2 & 5.3 & 9.0 &2.2& 4.8& 7.5\\
		Normal & 1.7 & 3.4 & 5.0 &1.6& 4.0& 6.5 \\
		Moderate drought & 1.5 &3.2 & 5.0&1.5&3.7& 5.7\\
		Severe drought, & 1.1 &2.5 &5.5&1.4& 3.3& 5.4\\
		Extreme drought & &&& 1.1 & 2.9   & 4.8         \\
		\bottomrule
	\end{tabular}
\end{table}

% \edit{Further graphs and tables are shown in the Supplementary Material, including: percentage standard deviation remaining for lead times of 1 to 10 weeks, see Fig.~\ref{fig:GP_NDVI_forecast} for GP forecasting on Landsat data and Fig.~\ref{fig:NDVI_forecast} for AR forecasting on MODIS data; the $R^2$-score and reduction in standard deviation for the two, four and six weeks forecasts for each individual region, see Tables \ref{tab:NDVI_LS} to \ref{tab:NDVI_MODIS}; tables of results of NDVI forecasting with the other dataset/method combinations, see Table \ref{tab:NDVI_GPM} for GP forecasting on MODIS data and Table \ref{tab:NDVI_Landsat_AR} for AR forecasting on Landsat data.}



\begin{figure}
	\centering
	\includegraphics[trim = 30mm 15mm 0mm 10mm,width=5.4 cm]{figures/seasonal_LANDSAT_1Region.pdf} \qquad \includegraphics[trim = 30mm 15mm 0mm 10mm,width=5.4 cm]{figures/seasonal_MODIS_1Region.pdf}
	\vspace{0.5cm}
	\caption{RMSE of VCI3M forecast for each week of the year. (a) GP forecasting on Landsat data. (b) AR forecasting on MODIS data. Grey shading indicate the rainy seasons, March-May and October-December.} \label{fig:seasonal}
\end{figure}

%{\color{red} Is the following paragraph more a combination of results and discussion? Plus, I am not sure that the seasonality plot links well with the gap filling as we have not looked at the seasonality of the gaps. The seasonality mainly reflects the facts that droughts are more important to forecast/monitor in the rainy seasons than in the dry season. Wonder whether the results from figure C6 from Suppl Mat should rather follow within the paragraph above on gap filling "Due to the presence of non-interpolated gaps" My 2 cents. Pedram}
The fact that seasonal differences in RMSE are not substantial provides reassurance that the forecast accuracy estimates are not inflated by the gap-filling during preprocessing. If this were the case, there would be a sustained drop in RMSE during the more overcast months of the year (March to May and October to December). While the GP forecasts on the Landsat data were computed from time series on which no future data were used for the interpolations (see \ref{sec:preprocessing}), for the MODIS data interpolations did make use of future data. Therefore, to obtain further reassurance that performance estimates of AR forecasting on the MODIS data were not inflated, a plot was made of RMSE at 4 weeks lead time against percentage of pixels from which a good observation was obtained on the date of the forecast, see Fig.~\ref{fig:percentclear} in the Supplementary Material. There was no apparent correlation (Pearson coefficient was 0.01), and hence it was concluded that the gap-filling was not leading to inflated forecast performance.

\subsection{Drought event forecast: ROC curves}



\begin{figure}%
	\centering
	\includegraphics[trim = 20mm 4mm 12mm 3mm,width=5.6 cm]{figures/VCIROC_Landsat.pdf}
	\qquad
	\includegraphics[trim = 12mm 4mm 20mm 3mm,width=5.6 cm]{figures/VCIROC_MODIS.pdf}
	\caption{(Left) ROC curve for drought detection (VCI3M $<$ 35) for lead times of 2, 4 and 6 weeks using the GP method on Landsat data. (Right) ROC curve for drought detection using the AR method on MODIS data. The curves are plotted from applying different thresholds to convert the continuous forecast into a binary forecast of drought or no drought, see text for details. The shaded circles show the point obtained from forecasting drought when the predicted VCI3M$<$35. The area under the curve is 1.0, 0.98, 0.96 (GP, left) and 1.0, 0.99 and 0.96 (AR, right) for lead times of two, four and six weeks, respectively. }
	\label{fig:ROC_abb}
\end{figure}





To assess the usefulness of the AR and GP methods for drought forecasting, we tested their ability to detect specific drought events, as defined by the NDMA's alert threshold \citep[VCI3M$<$35,][]{rs8040267}. ROC curves were plotted for the detection of VCI3M$<$35 at lead times of two, four and six weeks (Fig.~\ref{fig:ROC_abb}). These curves show the probability of predicting a state of drought (VCI3M$<$35) when there will be a state of drought, i.e.~hit rate, against the probability of predicting drought when there will not be drought, i.e.~false alarm rate, for varying binarisation thresholds on the forecast. These curves give an indication that one can reliably forecast droughts with these methods even as far as six weeks ahead. 

The ROC curve performance is not highly dependent on the region (see Table \ref{tab:ROC2}). Even for the wetter Eastern regions, for which observations are sparser due to cloud cover, the hit and false alarm rates only differ by 1 to 2 percentage points compared with those computed across all regions. Further, ROC curves for predicting the NDMA drought categories of severe (10$<$VCI3M$<$20) or extreme (VCI$<$10) drought look similar to those for detecting VCI3M$<$35, see Fig.~\ref{fig:ROCotherdrought} in the Supplementary Material. 

\begin{table}
	\small
	\caption{False alarm rate and hit rate (respectively, expressed in percent) for different regions in Kenya and at different lead times. This is based on forecasting drought if the predicted VCI3M is less than 35 (different performances could be obtained with different warning thresholds (see Figure \protect\ref{fig:ROC_abb}. Regions are composed of the following zones: North -- Z1,3 and 5; East --  Z7, 9, 10 and 11 and South -- (Z15 and 18))} \label{tab:ROC2}
	\centering
	\begin{tabular}{l|ccc|ccc} 
		\toprule
		\textbf{Regions} & & \textbf{Landsat GP} & & &\textbf{MODIS AR} \\
		& \textbf{2} & \textbf{4} & \textbf{6} & \textbf{2} & \textbf{4} & \textbf{6} \\
			& \multicolumn{3}{c|}{\textbf{weeks }}& \multicolumn{3}{c}{\textbf{weeks }}\\

		\midrule
%		All & 2\% \; 96\% & 4\% \; 87\% & 5\% \; 78\% & 2\% \; 97\% & 4\% \; 91\% & \; 7\% \; 84\%\\
%		Z24 & 2\% \; 99\% & 4\% \; 91\% & 5\%  \; 82\% & 2\% \; 98\% & 5\% \; 94\% & \; 8\%  \; 88\%\\
%		North  & 1\% \; 97\% & 2\% \; 88\% & 3\%  \; 76\% & 2\% \; 98\% & 6\% \; 93\% & 11\% \; 87\%\\
%		East  & 3\% \; 94\% & 5\% \; 85\% & 6\% \; 77\% & 3\% \; 97\% & 6\% \; 91\% & 10\% \; 85\%\\
%		South   & 1\% \; 96\% & 3\% \; 88\% & 4\% \; 77\% & 2\% \; 98\% & 6\% \; 94\% & 11\% \; 90\%\\
		All & 2 \; 96 & 4 \; 87 & 5 \; 78 & 2 \; 97 & 4 \; 91 & \; 7 \; 84\\
		Z24 & 2 \; 99 & 4 \; 91 & 5  \; 82 & 2 \; 98 & 5 \; 94 & \; 8  \; 88\\
		North  & 1 \; 97 & 2 \; 88 & 3  \; 76 & 2 \; 98 & 6 \; 93 & 11 \; 87\\
		East  & 3 \; 94 & 5 \; 85 & 6 \; 77 & 3 \; 97 & 6 \; 91 & 10 \; 85\\
		South   & 1 \; 96 & 3 \; 88 & 4 \; 77 & 2 \; 98 & 6 \; 94 & 11 \; 90\\
		\bottomrule
	\end{tabular}
\end{table}

 



%%%%%%%%%%%%%%%%%%%%%%%%%%%%%%%%%%%%%%%%%%
\section{Discussion} \label{sec:dis}
\linelabel{review:dis}
Droughts are complex and hence inherently difficult to define and measure \citep{MISHRA2010202}. A large number of satellite-based indicators have been developed to identify meteorological, hydrological, and agricultural droughts \citep{zargar2011review,aghakouchak2015remote} with each performing well in space and time to a certain degree \citep{ZHANG201796}. This paper uses two machine-learning methods to provide short-term forecasts of two satellite-based drought indicators, the weekly NDVI and the 3-month VCI (VCI3M) which are used by Kenya's National Drought Management Authority (NDMA) in their drought Early Warning System (EWS). We have investigated the skill and robustness of our forecasts in a number of ways.
%The linear autoregression models applied to MODIS achieved an $R^2$-score of 0.58 for NDVI anomaly at a lead time of 4 weeks, and an $R^2$-score of 0.95 for the VCI3M. The Gaussian Processes applied to Landsat achieved an $R^2$-score of 0.36 for NDVI anomaly at a lead time of 4 weeks, and an $R^2$-score of 0.94 for the VCI3M. 
Both of our methods showed high sensitivity and specificity for prediction of VCI3M values indicative of drought, at lead times of 2, 4 and 6 weeks (see Fig.~\ref{fig:ROC_abb}). They also perform much better than a persistence forecast (a factor of two in RMSE for VCI3M) and considerably better than a similar study, \cite{Adede2019}, that used a Artificial Neural Network model to predict future VCI for four Kenyan counties, and showed $R^2$-scores of 0.78 for a 1-month VCI3M forecast compared to 0.95 and 0.94 for our methods.
Moreover, these two methods provide robust results with either dataset (MODIS and Landsat), and are not impacted by the preprocessing steps. 
Finally, the methods present a high skill in forecasting drought irrespective of the region, the drought category, and the season.

Compared to other forecasting studies for pastures \citep{matere2019predictive,Adede2019,Papagiannopoulou,meroni2014early}, our methods perform better, provide useful forecasts with detailed skill assessments, and are rather simple as they only rely on one data stream as input and output. Furthermore, this data stream, the VCI3M, is already used (without forecast) by the NDMA and is thus a natural first step towards more ambitious and useful forecasts. Finally, the forecasts are reliant only on observations from satellites, which are available globally. The methods can thus be applied everywhere, providing there is sufficient calibration data.

A very important strength of our methods is the high level of skill. This may be surprising since the methods are rather simple and do not include other variables (e.g., rainfall, precipitation). \label{review:correlation}\linelabel{review:correlationL}
However, the natural growth cycles of vegetation and their response to environmental factors introduce temporal correlations (persistence) in the indices which can be exploited in short-range forecasting. In addition because as we are using the indicators themselves to determine the forecast we track all the factors that change the vegetation, e.g. disease, soil memory and land-use change and not just meteorological factors. This is confirmed as our skill did not improve much, when adding meteorological variables. However, future work should investigate the impact on our methods, in terms of skill and lead times, by including weekly, monthly and seasonal rainfall and temperature forecasts. These forecasts would allow to incorporate an estimate of exceptional seasonal events which will substantial impact on pastures.

Often, new forecast information developed by scientists to help the development and humanitarian sectors enhance disaster preparedness and response goes unused due to a ``usability gap'' between knowledge producers and users \citep{lemos2012narrowing}. In our study, the focus on VCI3M was mainly driven by the fact that this indicator is currently used by the NDMA to classify drought severity in the arid and semi-arid regions of Kenya. Such co-production strategies allow us to bridge the usability gap \citep{dilling2011creating,lemos2018naturesus} and provides confidence that our forecasting methods may be used.

We have also concentrated on methods that produce accurate short-term forecasts, rather than less-certain, but longer-range forecasts. We can speculate that while the latter might have greater value, the former might be more readily adopted in the monthly county bulletins released by the NDMA. Indeed, the forecasts developed here could, for example, help establish a new drought phase classification (`Early Alert') which, along with adequate preparedness actions developed by the disaster risk managers, would minimise the risk of a worsening drought condition. Anticipatory drought management strategies based on this `Early Alert' could for example focus on livestock vaccination programmes, livestock movement monitoring, or the repair of strategic water sources which enhance the resilience of these communities before a drought hits.

Forecasts alone do not necessarily lead to good anticipatory actions. Whilst acting ahead of disasters is on average more financially effective than responding to an event \citep{venton2012economics}, traditionally the humanitarian agencies tend to respond to disasters as financial resources are only available during or after an event. Additionally, due to the uncertainty in the models, anticipatory actions based on such forecasts do raise the risk of "acting in vain", which may have substantial negative impact on the humanitarian sector in the short term \citep{lopez2018bridging}. These agencies thus need access to adequate financial resources, e.g. forecast-based finance \citep[FbF,][]{nhess-15-895-2015} to fund anticipatory actions based on skill-assessed forecast in order to factor in the possible negative consequences of acting in vain. For the forecast methods developed in this study, the chance of acting in vain will be low due to the high level of skill, which will ultimately lower the barriers to uptake.




\section{Caveats and Future Work}
 \label{sec:futurework}
 \linelabel{review:futurework}

As discussed above, our methods are already suitably skillful that they are usable as they stand, however, there are some minor limitations and improvements that we have identified to enhance the functionality, skill, lead-time and impact of our forecasts. 


 \linelabel{review:pixels}
Our analysis has been based on relatively small samples of the available pixels, aggregated, spatially, at the level of livelihood zone and county intersections. This limits the localisation specificity of our predictions and reduces the accuracy (different land usages will have been merged).  The processing of all pixels can be achieved within reasonable computational constrains. This we will allow us to aggregate with localised regions of the same land-use classes which will provide greater accurach  suitable spatial scale (e.g., grazing units). 


 Our forecasts will be unavailable or less accurate in periods during or following cloud-cover gaps. More subtly, our validation will have favoured dry season observations, which are less affected by cloud cover, and this will have an impact on  the validation of the forecast performance. However, we found little variation in performance throughout the seasons so we do not think these are significant problems. 
 
 Drought indicators derived from radar satellites, such as Sentinel 1, could also be used to  reduce the impact of cloud cover.
 
 
% %
% Thirdly, the forecast, and its estimated skill, are only appropriate for the types of vegetation and environments for which it has been calibrated.
% %
% Fourthly, the vegetative indicators we have chosen may not be a good indicator of socio-economic drought. Finally, and most-significantly, our methods are only appropriate for relatively short lead-times. However, even 4-week lead times can be useful. 


% We have identified a number of elements of future work that can address some of the weaknesses of our methods.  

% \begin{itemize}
%     \item  provision of error estimates that are tailored for the specific conditions and data availability. 
%     \item  
 
%  \item t
%  \item 
 
%  but if enough data is available new models could be calibrated.  The skills in new regions will be different, but the similarity in quality we obtained in different regions of Kenya gives us some confidence that the skills will not be vastly different. 
 
%  \item drought indicators
%  this still needs some further assessment \citep{zargar2011review,aghakouchak2015remote,ZHANG201796}.
%  \item
%  , our method does not include other data (e.g., rainfall, temperature, LST). We expect forecast performance to increase if other variables such as rainfall and temperature were included. Climate factors, such as ENSO, may even allow us to reduce forecast errors and extend our lead times. 
% \item
% Future work should explore how long a temporal baseline is required for good calibration. 
% \itemIn the current implementation a single uncertainty is available for each region, but future work can extend this to an uncertainty estimate on every forecast. 


%  further strengthen EWS, future research needs to improve forecast skill of drought impact indicators with clearly-defined triggers and thresholds (which may vary in time and space). Additionally, financial systems need to be established that can be accessed based on such forecasts to be able to act across various timescales before the disaster occurs.

% \end{itemize}

\section{Conclusion}

In conclusion we have developed two new forecasting methods which provide similar skillful, short-range forecasts of vegetation indexes. The choice of input data, output indicators and demonstrated skill argues that these methods will be useful for drought early warning systems. Despite the limitations, there is clear evidence here of a statistical persistence model providing strong skill over useful lead times. This can be an important contribution to anticipatory drought risk management in Kenya

To further strengthen EWS, future research needs to improve forecast skill of drought impact indicators with clearly-defined triggers (e.g., threshold values based on forecasts, which may vary in time and space). Adequate policy and institutional arrangements are also needed to allow the various actors to engage and interact with a long-term perspective on risk management. This in turn, requires financial systems that can be accessed based on such forecasts to be able to act across various timescales before the disaster occurs (i.e. FbF).



\section*{Authors responsibilities}
A.B.B., S.D. and E.S. are lead authors as they contributed equally to the paper and the order of the three names is alphabetical. A.B.B was responsible for developing and running the AR method. S.D. was responsible for developing and running the GP method, and for the accumulation and processing of the Landsat data. E.S. was responsible for the MODIS data accumulation and preprocessing. JMw, SO and PR developed the initial idea and provided feedback throughout. All authors wrote, reviewed and edited the final manuscript. 

\section*{Acknowledgements}
This research was funded by the STFC through the following projects: ``AstroCast: Applying Astronomy Data Analysis to enhance disaster forecasting'' -- grant number ST/R004811/1; and ``STFC Official Development Assistance (ODA) Institutional Award" attached to the same grant; ``A UK-Africa Data Science Network: Capturing the SKA-Driven Data Transformation'' grant number ST/R001898/1; and by the Science for Humanitarian Emergencies and Resilience (SHEAR) consortium project ``Towards Forecast-based Preparedness Action'' (ForPAc, www.forpac.org), grant numbers NE/P000673/1. This project was initiated through pump-priming funding from the University of Sussex's ``Sussex Research'' thematic programme and carried out as part of the interdisciplinary Data Intensive Science Centre at the University of Sussex (DISCUS). We acknowledge early contributions to that pilot work from Peter Hurley, Philip Rooney, Martin Jung, and J\"{o}rn Scharlemann. Martin Todd and Alexander Antonarakis provided useful feedback which improved the manuscript.


%Test Table~SM~\ref{tab:sm1}\\
 %Figure SM~\ref{fig:interp_GP}


%\appendixtitles{no} %Leave argument "no" if all appendix headings stay EMPTY (then no dot is printed after "Appendix A"). If the appendix sections contain a heading then change the argument to "yes".
%\appendixsections{one} %Leave argument "multiple" if there are multiple sections. Then a counter is printed ("Appendix A"). If there is only one appendix section then change the argument to "one" and no counter is printed ("Appendix").
  
\section*{References}

\bibliography{mybibfile}

\newpage
  
  \renewcommand\appendixname{Supplementary Material }
\setcounter{figure}{0}    
\setcounter{table}{0}
\setcounter{page}{1}

\appendix

\section*{Supplementary Material}

\section{Data selection and comparison of datasets} \label{sec:datasets}

\subsection{Landsat}
Landsat-5, 7 and 8 \citep{royetal} red and near infrared (NIR) surface reflectances and quality assessment (QA) data over the 10 pastoral livelihood zones of Kenya, from January 1st, 2000 to February 1st, 2019, were obtained using the United States Geological Survey (USGS) EarthExplorer. Specifically, data from 1\,000 pixels within each region were drawn from the Level-1 Precision Terrain (L1TP) processed dataset, which has well-characterized radiometry and is inter-calibrated across the different Landsat sensors. The spatial resolution of these data is 30m and the repeat interval is 16 days. Using the QA data, observations classified as clear from clouds or cloud shadows were kept. Pixels with fewer than half of the observations over the full time period were discarded (and replaced with an alternative random selection, with a few exceptions, see Table \ref{tab:NDVI_LS}). %Landsat-5 data were available up until November 2011 (albeit with several large gaps), Landsat-7 data were available for the whole time-period, and Landsat-8 data were available from March 2013. 
The surface reflectances were combined to obtain 
NDVI. %\edit{\footnote{${\rm NDVI}=({\rm NIR}-{\rm Red})/({\rm NIR}+{\rm Red})$}} 
% from which we derived the NDVI anomaly and VCI {\color{blue}(see Eq. \ref{vci_eq})}. 
%, \edit{out of those for which at least half of the observations were present and labelled good quality.}\footnote{\edit{Except for a few regions, for which this threshold had to be dropped in order to be able to obtain 1\,000 pixels, see Table \ref{tab:NDVI_LS} in the Appendix.}} 

\subsection{MODIS}

NDVI data were also gathered from the surface reflectances obtained from the daily, 500-meter resolution MODIS Terra/Aqua Nadir BRDF-Adjusted Reflectance product \citep[MCD43A4,v006;][]{Schaaf2015}. Data from February 22nd, 2000 up to February 1st, 2019 were acquired via the NASA Land Processes Distributed Active Archive Center. QA maps files with binary quality flags were used to remove poor quality data resulting from cloud or unreliable BRDF corrections. Data were drawn from 100  pixels within each region, out of those that had been  identified  as grassland by the MODIS land cover classification maps (MCD12Q1,v006). 
% using Application for Extracting and Exploring Analysis Ready Samples (AppEEARS),
 %The spatial resolution of \edit{this product} is 500m and the repeat interval is 1 day, \edit{with the daily data being 16 day composites of raw observations at 1 to 2 day intervals}. 


\subsection{Comparison of the two datasets} 

\begin{table}[!hb]
	\small
	\caption{Table comparing Landsat and MODIS products}\label{tab:sm1}
	\label{tab:comp}
	\centering
	\begin{tabular}{p{2cm}p{4cm}p{4cm}}
		\toprule
		\textbf{Feature}	& \textbf{Landsat} & \textbf{MODIS}\\
		\midrule
		\textbf{Spatial \newline Resolution} & 	High resolution at 30\,m 	&   Medium resolution ranging from 250\,m to 1\,km \\
		\textbf{Temporal \newline Resolution}	&  16-day sampling (8-day when both Landsat-7 and 8 are used	&   Daily sampling  monitoring dynamic variables  \\ 
		\textbf{Quality} &Cloud coverage at 30\,m & Cloud coverage at 500\,m \\
		
		\bottomrule
	\end{tabular}
\end{table}

The key differences between the two datasets are the spatial and temporal resolutions, see Table \ref{tab:comp}. The Landsat data had higher spatial resolution, whilst the MODIS data had higher temporal resolution. Since forecasting was being attempted at the level of large scale regions (livelihood zone and county intersections), and at a weekly temporal resolution, the expectation was that the MODIS data would have advantages,  assuming individual Landsat and MODIS observations have similar signal-to-noise ratios. The processed MODIS time series with weekly observations have less measurement noise because they are composites of 7 daily observations (that themselves are 16-day composites of measurements taken every 1-2 days), whereas the processed Landsat time series are derived from more temporally sparse data (up to 3 different Landsat missions, each yielding one observation every 16 days). Landsat data would have advantages in different applications where forecasts on smaller spatial scales are required. The Landsat data also has the advantage that the quality flags and cloud masks are defined on smaller scales.    

The differences between the MODIS and Landsat datasets produced slightly different `true' aggregate time series on which to assess the interpolation and forecasting methods. In addition to the different temporal resolution of the observations supplying the final time series, the MODIS data were aggregated across 100 random grassland pixels from each region, whereas the 1\,000 Landsat pixels analysed were randomly distributed over the whole of each region. In choosing how many pixels to analyse per region, there is a trade-off between using a larger number of pixels for higher accuracy, and a smaller number of pixels for lower computational cost. Fewer MODIS pixels were used than Landsat pixels since they correspond to larger spatial regions. Both these choices of number of pixels should be sufficient for high accuracy of results, since for Landsat data the $R^2$-score comparing the average of all pixels from a region with the average of 100 or 1\,000 random pixels was 0.990 and 0.9993 respectively. The MODIS grassland classification was not available at Landsat resolution, thus unambiguous classification of the smaller Landsat pixels was not possible. This is unlikely to have made much difference to pixel selection, given that the pastoral livelihood zones are mostly grasslands (Fig. \ref{fig:l_zone}). 

\section{Further details on preprocessing}

\subsection{Gaussian process modelling} \label{sec:GPexplain}
A Gaussian Process is a probabilistic model defined as a collection of random variables for which any finite subset has a joint Gaussian distribution \citep{gpm}. Formally, for the present application of interpolation or extrapolation of a time series, with observation at time $t$ denoted by $X_t$, the model is

\begin{align}
    &X_t\sim \mathcal{N}\left[ Y(t),\sigma_r^2\right]\,,\\
    &Y(t) \sim \mathcal{GP}\left[ m(t),k(t,t^{\prime})\right]\,.
\end{align}
Here $Y(t)$ is the true value of the observed index, and the measurement noise is $\sigma_r$, so that an observation $X_t$ is a normal random variable with mean $Y(t)$ and standard deviation $\sigma_r$. The true values $Y(t)$ are also normally distributed, with the mean at time $t$ given by the mean function $m(t)$, and the covariance between values at times $t$ and $t^{\prime}$ given by the kernel function $k(t,t^{\prime})$. To carry out interpolation or extrapolation from a time series, existing data are used to fit the mean, $m$, kernel, $k$, and measurement noise $\sigma_r$, and then expected values are produced for the desired times, based on the obtained fit.

For gap-filling on individual Landsat pixel NDVI time series, the model was determined as follows, using the \texttt{Pyro} programming package for \texttt{Python}. The mean, $m(t)$, was assumed to be constant, and the mean of the whole time series. To determine the kernel, Compositional Kernel Search \citep{pmlr-v28-duvenaud13} was used. A search through products and sums of up to two common kernel combinations (Linear, Radial Basis Function, Periodic, Rational Quadratic and Matern) yielded Radial Basis Function (RBF) plus Periodic $(k_{\mathrm{RBF}} + k_{\mathrm{P}})$, with period $p$ set to one year, as the kernel to use:

\begin{align}
	k_{\rm{RBF}}(t,t') &= \sigma_{\rm{RBF}}^2 \exp{\Big( -0.5 \frac{|t-t'|^2}{l_{\rm{RBF}}^2} \Big) }\,,\\
	k_{\rm{P}}(t,t')&=\sigma_{\rm{P}}^2 \exp \left( -2 \frac{ \sin^2(\pi|t-t'|/p)}{l_{\mathrm{P}}^2} \right)\,.
\end{align}
There were thus 5 parameters to fit for each time series $(\sigma_r,\sigma_{\rm{RBF}},l_{\rm{RBF}},\sigma_{\rm{P}},l_{\mathrm{P}})$. These were learned using Stochastic Variational Inference (SVI).

For the forecasting on the aggregated NDVI anomaly and VCI3M, a pure Radial Basis Function kernel was used, since for these anomaly indices, the periodic component is not present, see Section \ref{sec:methods} in the main manuscript.


\subsection{Gap-filling for MODIS}\label{sec:Lmax}

Interpolation of gaps in the raw MODIS time series was not carried out when the length of the gap was longer than a certain maximum, $L_{\rm{max}}$. In choosing $L_{\rm{max}}$, a trade off between quality and quantity of remaining observations had to be made. The choice $L_{\rm{max}}=6\, \rm{weeks}$ was made, after exploring a range of values and finding results to be not sensitive to the precise choice within the range between 4 and 8 weeks, see Table \ref{tab:IL choices}. 
% This meant that all interpolated observations were no more than 3 weeks distant from a real observation, which is within the range for which interpolation can be assumed to be reasonably accurate, given the forecasting results found. 
Note that interpolation on the Landsat data was carried out for all gaps, since the GP interpolation method makes use of the entire time series, and interpolated values within a long interpolation take values close to the seasonal mean.

\begin{table}
	\caption{Comparison of outcomes for different choices of maximum allowed interpolation length $L_{\mathrm{max}}$ on the MODIS data. $R^2$-score of 4 week AR forecast and the percentage of the time that it was possible to make a forecast, for $L_{\mathrm{max}}=4$, 6, and 8 weeks. Numbers show the median across all regions.} \label{tab:IL choices}
	\centering
	%% \tablesize{} %% You can specify the fontsize here, e.g.,  \tablesize{\footnotesize}. If commented out \small will be used.
	\begin{tabular}{ccc} 
		\toprule
		\textbf{$L_{\mathrm{max}}$ (weeks)}   & \textbf{$R^2$-score}    & \textbf{Forecasts attempted (\%)}\\
		\midrule
		4  & 0.60 & 84 \\
		6  & 0.58 & 93 \\
		8 & 0.63 & 98 \\
		\bottomrule
	\end{tabular}
\end{table}

Due to the presence of non-interpolated gaps in the MODIS time series, there were weeks when a forecast assessment was not carried out on these data. The criteria for being able to do AR forecasting on these data were: (i) the three most recent weekly aggregated observations had to be present, since these are required for making a prediction; (ii) there had to be an aggregated observation present for the week being forecast, so the quality of the prediction could be assessed.\footnote{GP forecasting was still possible when (i) failed, but was also not carried out in that case, since performance would have been worse than usual in this case.}


\subsection{Comparison of other possible gap-filling methods} \label{sec:compare}
Various gap-filling methods have been used to deal with missing values resulting from the presence of clouds and atmospheric aerosols. These methods are based on either spatial information, temporal information or some combination of both spatial and temporal information \citep{bg-10-4055-2013, Weiss2014}. Temporal interpolation was chosen given that spatial interpolation methods suffer from the fact that there are frequently clouds over Kenya that cover large groups of neighbouring pixels (although a possible alternative, not considered here, would be to make use of other pixels that historically behave similarly in time \citep{Cao:2018}).

The performance of the temporal gap-filling methods employed, compared with alternative temporal gap-filling methods, was tested by removing observations, applying the method, and then comparing the interpolated observations with the removed observations. GP interpolation and linear, quadratic and cubic polynomial interpolation methods were tested, on both the Landsat and MODIS datasets. $R^2$-scores were obtained for using the interpolated values to predict the `true' values for the missing observations.

% An example of the pixel level interpolation on Landsat data can be found in Appendix \ref{fig:interp}. 

For the Landsat data, one randomly chosen observation between 1/1/2014 and 1/2/2019 was removed from each of 2000 randomly selected individual pixel time series.
% \footnote{We remove the mean of the individual NDVI time series for every single observed and interpolated datum before calculating the $R^2$-scores. This avoids an over-estimate of the denominator (see Equation \ref{eq:eqr}) due to the variance from different regions in Kenya. This also forces the mean value prediction to be zero, which it should be for a $R^2$ calculation.} 
From the MODIS data, 2000 random individual pixel NDVI time series (1/1/2014 to 1/2/2019) were chosen. 20 randomly selected NDVI values were dropped from each of the time series and the various gap-filling methods were used to interpolate the dropped values. The results for Landsat are shown in Table \ref{tab:comp_int}, and for MODIS in Table \ref{tab:comp_int2}. Note that with these methods, the random samples are more likely to come from periods when there are not many gaps. It is an assumption that these methods are still the correct ones to choose during periods when there are lots of gaps.



\begin{table}
	\caption{Comparison of GP method with commonly used interpolation methods as candidates for gap-filling on Landsat data. At the pixel level a random observation was removed, and then interpolated with each of the listed methods.} \label{tab:comp_int}
	\centering
	%% \tablesize{} %% You can specify the fontsize here, e.g.,  \tablesize{\footnotesize}. If commented out \small will be used.
	\begin{tabular}{lc} 
		\toprule
		\textbf{Method}  & \textbf{$R^2$-score} \\
		\midrule
		GP & 0.67 \\
		Linear & 0.53 \\
		Quadratic & -0.07 \\
		Cubic & -1.92 \\
		Last value & 0.34 \\
		Mean value & 0.0 \\
		\bottomrule
	\end{tabular}
\end{table}

For the Landsat data, the GP method achieved the highest $R^2$-score, thus showing its utility, and justifying our choosing it. The $R^2$-score of 0.67, achieved by the GP method, is close to the $R^2$-score of 0.76 which is obtained from using one Landsat observation to predict another Landsat observation from the same 16-day observation period (which could be checked, since there were instances of duplicate observations due to overlap of tiles). Fig.~\ref{fig:interp_GP} shows a contour plot of the true versus interpolated NDVI observations using this method. This plot shows that the method doesn't introduce any biases- the slope and intercept are approximately 1 and 0 respectively.

\begin{figure} 
	\centering
	\includegraphics[trim = 30mm 5mm 5mm 15mm,width=8cm]{figures/interp_GP.pdf}
	\caption{Contour plot of Landsat observed and predicted NDVI values from the GP interpolation.} \label{fig:interp_GP}
\end{figure}

%For interpolation the linear method was also somewhat effective, achieving an $R^2$-score of 0.53. 

% When SG smoothing was applied (windows of length 7 and a polynomial of order 2) after the linear interpolation, the $R^2$-score rose to 0.60, which was not quite as high as that obtained from the GP method. When doing (forecast mode) extrapolation the alternatives to the GP method did not work at all ($R^2$-scores were negative). 



For the MODIS data, GP, linear interpolation and quadratic interpolation all performed similarly well. Quadratic interpolation had the highest $R^2$-score, hence this method was chosen for gap-filling on the MODIS data. The higher interpolation $R^2$-scores for MODIS, compared to Landsat, imply that the MODIS data is less noisy than the Landsat data. Assuming that observations from MODIS and Landsat have similar signal-to-noise ratio, this can be explained by the higher temporal resolution of MODIS, and the compositing of multiple observations for the weekly gridded MODIS data. Fig.~\ref{fig:interpScatter} shows a contour plot of the true versus interpolated NDVI observations using the quadratic interpolation method. This again demonstrates that the interpolation doesn't introduce biases- the slope and intercept are approximately 1 and 0 respectively.

\begin{table}
	\caption{Comparison of interpolation methods as candidates for gap-filling on MODIS data.
% 	The interpolation score column is the mean $R^2$-score for all sampled (1200) time series.
	} \label{tab:comp_int2}
	\centering
	\begin{tabular}{lc} 
		\toprule
		\textbf{Method}  & \textbf{$R^2$-score} \\
		\midrule
		GP & 0.92 \\
		Linear & 0.93\\
		Quadratic & 0.94\\
		Cubic & 0.92\\
		Last value & 0.70\\
		Mean value & -0.02\\
		\bottomrule
	\end{tabular}
\end{table}
\vspace{2cm}
\begin{figure}[H]
	\centering
	\includegraphics[trim = 30mm 5mm 5mm 15mm,width=8cm]{figures/QuadraticInt_contour.pdf} 
	\caption{Contour plot of MODIS observed and predicted NDVI values from 2000 pixels for gap-filling by quadratic interpolation.} \label{fig:interpScatter}
\end{figure}

% For the MODIS data, forecasting was performed between 1/1/2004 and 1/2/2019 by removing all future data (and also removing the data from the past 1 to 10 weeks), creating a forecast mode GP and comparing the resulting forecast observations with the corresponding removed observations.\footnote{The forecast was only made for the dates for which AR forecasting was attempted (see Section \ref{sec:ARmethod} for description of exclusions).} 





% \begin{figure} 
% 	\centering
% 	\includegraphics[trim = 35mm 0mm 0mm 0mm,width=12 cm]{figures/interp.pdf} 
% 	\caption{An example of our interpolation method on a Landsat pixel, the black crosses show the real data, where the datum at day 6698 is removed. The colored dots show our interpolated value at day 6698 using the different methods. The blue, red, green and yellow lines show the complete interpolation for every date. } \label{fig:interp}
% \end{figure}


% \begin{figure}
% 	\centering
% 	\includegraphics[trim = 35mm 0mm 0mm 0mm,width=12 cm]{figures/interp_2.pdf} 
% 	\caption{An example of interpolation methods on MODIS pixels. The blue cross shows the observation at day 5403, which was then cut and interpolated using different methods. The colored dots show interpolated values using the different methods.  } \label{fig:interp2}
% \end{figure}



\newpage

\section{Further forecast results}

Figs.~\ref{fig:GP_NDVI_forecast} and \ref{fig:NDVI_forecast} plot the forecast performance of the two methods in terms of percentage of standard deviation remaining $S$, for lead times of 1 to 10 weeks. For NDVI anomaly, for both methods, $S$ approaches the baseline of 100 as the lead time approaches 10 weeks, while for VCI3M, some forecast skill is still apparent at a lead time of 10 weeks. Fig.~\ref{fig:persistence} compares the performance of the AR VCI3M forecast with that of the persistence VCI3M forecast, on the MODIS data; the persistence forecast being simply the most recent observation. The AR forecast performs substantially better than the persistence forecast, for example, achieving a RMSE of approximately half that of the persistence forecast for a lead time of 4 weeks. The GP VCI3M forecast on the Landsat data achieves a similar improvement on the persistence forecast. Fig.~\ref{fig:percentclear} shows, for the MODIS/AR method, the average RMSE of a 4 week forecast against the percentage of pixels from which there was a clear observation during the week the forecast was made. Fig.~\ref{fig:ROCotherdrought} shows alternative ROC curves for drought prediction using the AR method on the MODIS data, based on different thresholds for defining drought.

\begin{figure}[H]
	\centering
	\includegraphics[trim = 50mm 0mm 0mm 0mm,width=14 cm]{figures/FC-pref.pdf} 
	\caption{Forecast performance with a lead time of 1 to 10 weeks using the GP method on the Landsat data, as given by percentage standard deviation remaining $S$, for (Left) NDVI anomaly, and (Right) VCI3M. The blue lines show results for the individual regions (county/livelihood zone intersections), and the black line shows the median across all regions.} \label{fig:GP_NDVI_forecast}
\end{figure}

\begin{figure}[H]
	\centering
	\includegraphics[trim = 20mm 0mm 0mm 0mm,width=12.5cm]{figures/NDVI_forecast2.pdf} 
	\caption{Forecast performance with a lead time of 1 to 10 weeks using the AR method on the MODIS data, as given by percentage standard deviation remaining, for (Left) NDVI anomaly, and (Right) VCI3M. The blue lines show results for the individual regions for which a forecast is possible more than 50\% of the time, and the black line shows the median across all 19 of these regions.} \label{fig:NDVI_forecast}
\end{figure}

\begin{figure}[H]
	\centering
	\includegraphics[trim = 20mm 0mm 0mm 0mm,width=12.5cm]{figures/persistence.pdf} 
	\caption{Comparison of AR forecast with persistence forecast on the MODIS data. For lead times of 1 to 10 weeks, the RMSE of the AR forecast as a percentage of the RMSE of the persistence forecast. The blue lines show results for the individual regions for which a 4 week forecast is possible more than 50\% of the time, and the black line shows the median across these regions.} \label{fig:persistence}
\end{figure}

\begin{figure}[H]
	\centering
	\includegraphics[trim = 30mm 35mm 0mm 10mm,width=10 cm]{figures/ngaps.pdf} 
	\vspace{0.8cm}
	\caption{RMSE of 4 week forecast against percentage of clear pixels at most recent observation, for the AR method on the MODIS data. Plotted points are RMSE for each integer percentage of clear pixels. The Pearson correlation here is 0.01.} \label{fig:percentclear}
\end{figure}

\newpage

\begin{figure}[H]
	\centering
%	\includegraphics[trim = 30mm 35mm 0mm 10mm,width=5.4 cm]{figures/VCIotherROCabb.pdf} \qquad \includegraphics[trim = 0mm 35mm 30mm 10mm,width=5.4 cm]{figures/VCIotherROCabb.pdf}
\includegraphics[trim = 30mm 35mm 0mm 10mm,width=5.4cm cm]{figures/VCIotherROCabb.pdf} 
	\caption{ROC curves for predicting drought with drought defined at various NDMA thresholds. Possible hit rates against possible false alarm rates for the AR method on the MODIS data for the detection of: (Top) Any drought, VCI3M$<$35, (Middle) Severe or extreme drought VCI3M$<$20, (Bottom) Extreme drought VCI3M$<$10.} \label{fig:ROCotherdrought}
\end{figure}

\newpage
\subsection{Tables of NDVI anomaly and VCI3M forecast performance by region} \label{app}

% \begin{table}[H]
% 	\footnotesize
% 	\caption{NDVI anomaly forecast using Landsat data for the 29 regions. The numbers shown are the proportion of standard deviation remaining (Equation \ref{eq:S}) and the $R^2$-score for NDVI anomaly. We only used past data for the interpolation and we the average value for every pixel within the region for the region estimate. The * indicates regions where a minimum of 180 detections per pixel where used, instead of 250. } \label{tab:NDVI_LS}
% 	\centering
% 	\begin{tabular}{l|ccc} 
% 		\toprule
% 		\textbf{Region}   & \textbf{2 weeks}  & \textbf{4 weeks}  & \textbf{6 weeks} \\
% 		\midrule
% 		Baringo Z24 & 46 0.74 & 66 0.46 & 81 0.19 \\
% 		Elgeyo-Marakwet Z24 & 49 0.74 & 69 0.47 & 83 0.22 \\
% 		Garissa Z10* & 55 0.64 & 73 0.36 & 86 0.12 \\
% 		Garissa Z11* & 63 0.58 & 80 0.33 & 89 0.16 \\
% 		Isiolo Z5 & 57 0.64 & 75 0.37 & 88 0.13 \\
% 		Isiolo Z9 & 65 0.53 & 79 0.30 & 89 0.13 \\
% 		Isiolo Z10 & 79 0.31 & 91 0.08 & 96 -0.02 \\
% 		Isiolo Z24 & 57 0.67 & 76 0.41 & 89 0.19 \\
% 		Kajiado Z15 & 45 0.76 & 63 0.52 & 78 0.28 \\
% 		Kajiado Z18* & 44 0.75 & 59 0.55 & 72 0.34 \\
% 		Laikipia Z24 & 42 0.82 & 61 0.62 & 77 0.39 \\
% 		Lamu Z11* & 80 0.33 & 92 0.12 & 95 0.07 \\
% 		Mandera Z7 & 76 0.25 & 88 0.00 & 93 -0.13 \\
% 		Mandera Z9 & 53 0.44 & 69 0.06 & 80 -0.27 \\
% 		Marsabit Z5 & 52 0.60 & 66 0.35 & 78 0.11 \\
% 		Marsabit Z7* & 47 0.76 & 62 0.59 & 74 0.40 \\
% 		Narok Z15 & 56 0.67 & 80 0.34 & 92 0.12 \\
% 		Narok Z18 & 56 0.68 & 75 0.42 & 88 0.20 \\
% 		Samburu Z5 & 49 0.68 & 69 0.36 & 84 0.08 \\
% 		Samburu Z24 & 45 0.78 & 65 0.54 & 81 0.30 \\
% 		Tana River Z11* & 65 0.57 & 81 0.33 & 91 0.15 \\
% 		Turkana Z1 & 54 0.56 & 72 0.21 & 84 -0.09 \\
% 		Turkana Z3 & 38 0.61 & 51 0.33 & 60 0.06 \\
% 		Turkana Z24 & 46 0.75 & 66 0.48 & 81 0.21 \\
% 		Wajir Z7* & 48 0.71 & 62 0.51 & 74 0.30 \\
% 		Wajir Z9 & 79 0.20 & 91 -0.07 & 95 -0.18 \\
% 		Wajir Z10 & 87 0.24 & 99 0.01 & 100 0.00 \\
% 		WestPokot Z1 & 50 0.69 & 69 0.43 & 83 0.18 \\
% 		WestPokot Z24 & 49 0.68 & 66 0.42 & 79 0.16 \\
% 		\bottomrule
% 		Median & 53 0.67 & 69 0.36 & 84 0.15 \\
% 		\bottomrule
% 	\end{tabular}
% \end{table}

\begin{table}[H]
	\footnotesize
	\caption{NDVI anomaly forecast performance ($R^2$-scores)  using the GP method on Landsat data for the 29 regions. Asterisks indicate regions where a minimum of 180 detections per pixel where used, instead of 250. } \label{tab:NDVI_LS}
	\centering
	\begin{tabular}{l|ccc} 
		\toprule
		\textbf{Region}   & \textbf{2 weeks}  & \textbf{4 weeks}  & \textbf{6 weeks} \\
		\midrule
		Baringo Z24 &  0.74 &  0.46 & 0.19 \\
		Elgeyo-Marakwet Z24 & 0.74 &  0.47 & 0.22 \\
		Garissa Z10* &  0.64 &  0.36 & 0.12 \\
		Garissa Z11* &  0.58 &  0.33 & 0.16 \\
		Isiolo Z5 &  0.64 &  0.37 & 0.13 \\
		Isiolo Z9 &  0.53 &  0.30 & 0.13 \\
		Isiolo Z10 &  0.31 &  0.08 & -0.02 \\
		Isiolo Z24 &  0.67 &  0.41 & 0.19 \\
		Kajiado Z15 &  0.76 &  0.52 & 0.28 \\
		Kajiado Z18* &  0.75 &  0.55 & 0.34 \\
		Laikipia Z24 &  0.82 &  0.62 & 0.39 \\
		Lamu Z11* &  0.33 &  0.12 & 0.07 \\
		Mandera Z7 &  0.25 &  0.00 & -0.13 \\
		Mandera Z9 &  0.44 &  0.06 & -0.27 \\
		Marsabit Z5 &  0.60 &  0.35 & 0.11 \\
		Marsabit Z7* &  0.76 &  0.59 & 0.40 \\
		Narok Z15 &  0.67 &  0.34 & 0.12 \\
		Narok Z18 &  0.68 &  0.42 & 0.20 \\
		Samburu Z5 &  0.68 &  0.36 & 0.08 \\
		Samburu Z24 &  0.78 &  0.54 & 0.30 \\
		Tana River Z11* &  0.57 &  0.33 & 0.15 \\
		Turkana Z1 &  0.56 &  0.21 & -0.09 \\
		Turkana Z3 &  0.61 &  0.33 & 0.06 \\
		Turkana Z24 &  0.75 &  0.48 & 0.21 \\
		Wajir Z7* &  0.71 &  0.51 & 0.30 \\
		Wajir Z9 &  0.20 &  -0.07 & -0.18 \\
		Wajir Z10 &  0.24 &  0.01 &0 0.00 \\
		WestPokot Z1 &  0.69 &  0.43 & 0.18 \\
		WestPokot Z24 &  0.68 &  0.42 & 0.16 \\
		\bottomrule
		Median &  0.67 &  0.36 & 0.15 \\
		\bottomrule
	\end{tabular}
\end{table}


% \begin{table}
% 	\footnotesize
% 	\caption{VCI3M forecast performance using GPs on the Landsat data. The numbers shown are the percentage standard deviation remaining and the $R^2$ score, respectively.} \label{tab:VCI_LS}
% 	\centering
% 	%% \tablesize{} %% You can specify the fontsize here, e.g.,  \tablesize{\footnotesize}. If commented out \small will be used.
% 	\begin{tabular}{l|ccc} 
% 		\toprule
% 		\textbf{Region}  &  \textbf{2 weeks} &  \textbf{4 weeks}  & \textbf{6 weeks}  \\
% 		\midrule
% 		Baringo Z24 & 9 0.99 & 22 0.95 & 38 0.86 \\
% 		Elgeyo-Marakwet Z24 & 9 0.99 & 21 0.96 & 36 0.87 \\
% 		Garissa Z10 & 10 0.99 & 23 0.95 & 39 0.85 \\
% 		Garissa Z11 & 11 0.99 & 25 0.94 & 41 0.83 \\
% 		Isiolo Z5 & 10 0.99 & 23 0.95 & 39 0.85 \\
% 		Isiolo Z9 & 11 0.99 & 24 0.94 & 38 0.85 \\
% 		Isiolo Z10 & 13 0.98 & 29 0.92 & 46 0.79 \\
% 		Isiolo Z24 & 10 0.99 & 23 0.95 & 39 0.85 \\
% 		Kajiado Z15 & 9 0.99 & 21 0.96 & 36 0.87 \\
% 		Kajiado Z18 & 9 0.99 & 20 0.96 & 34 0.88 \\
% 		Laikipia Z24 & 7 0.99 & 18 0.97 & 32 0.89 \\
% 		Lamu Z11 & 13 0.98 & 29 0.92 & 45 0.80 \\
% 		Mandera Z7 & 15 0.98 & 32 0.90 & 49 0.76 \\
% 		Mandera Z9 & 12 0.98 & 29 0.92 & 48 0.77 \\
% 		Marsabit Z5 & 11 0.99 & 25 0.94 & 41 0.83 \\
% 		Marsabit Z7 & 8 0.99 & 19 0.96 & 32 0.90 \\
% 		Narok Z15 & 10 0.99 & 25 0.94 & 41 0.83 \\
% 		Narok Z18 & 11 0.99 & 24 0.94 & 40 0.84 \\
% 		Samburu Z5 & 10 0.99 & 24 0.94 & 42 0.83 \\
% 		Samburu Z24 & 8 0.99 & 20 0.96 & 35 0.88 \\
% 		TanaRiver Z11 & 11 0.99 & 24 0.94 & 40 0.84 \\
% 		Turkana Z1 & 14 0.98 & 31 0.90 & 52 0.73 \\
% 		Turkana Z3 & 12 0.99 & 26 0.93 & 43 0.81 \\
% 		Turkana Z24 & 9 0.99 & 22 0.95 & 38 0.85 \\
% 		Wajir Z7 & 9 0.99 & 20 0.96 & 34 0.88 \\
% 		Wajir Z9 & 16 0.97 & 35 0.88 & 53 0.72 \\
% 		Wajir Z10 & 17 0.97 & 35 0.88 & 52 0.73 \\
% 		WestPokot Z1 & 9 0.99 & 23 0.95 & 39 0.85 \\
% 		WestPokot Z24 & 10 0.99 & 22 0.95 & 38 0.85 \\
% 		\bottomrule
% 		Median & 10 0.99 & 24 0.94 & 39 0.85 \\
% 		\bottomrule
% 	\end{tabular}
% \end{table}

\begin{table}
	\footnotesize
	\caption{VCI3M forecast performance ($R^2$-scores) using the GP method on Landsat data.} \label{tab:VCI_LS}
	\centering
	%% \tablesize{} %% You can specify the fontsize here, e.g.,  \tablesize{\footnotesize}. If commented out \small will be used.
	\begin{tabular}{l|ccc} 
		\toprule
		\textbf{Region}  &  \textbf{2 weeks} &  \textbf{4 weeks}  & \textbf{6 weeks}  \\
		\midrule
		Baringo Z24 & 0.99 & 0.95 & 0.86 \\
		Elgeyo-Marakwet Z24 & 0.99 & 0.96 & 0.87 \\
		Garissa Z10 &  0.99 & 0.95 & 0.85 \\
		Garissa Z11 &  0.99 & 0.94 & 0.83 \\
		Isiolo Z5 &  0.99 & 0.95 & 0.85 \\
		Isiolo Z9 &  0.99 & 0.94 & 0.85 \\
		Isiolo Z10 &  0.98 & 0.92 & 0.79 \\
		Isiolo Z24 &  0.99 & 0.95 & 0.85 \\
		Kajiado Z15 & 0.99 & 0.96 & 0.87 \\
		Kajiado Z18 & 0.99 & 0.96 & 0.88 \\
		Laikipia Z24 & 0.99 & 0.97 & 0.89 \\
		Lamu Z11 &  0.98 & 0.92 & 0.80 \\
		Mandera Z7 &  0.98 & 0.90 & 0.76 \\
		Mandera Z9 &  0.98 & 0.92 & 0.77 \\
		Marsabit Z5 &  0.99 & 0.94 & 0.83 \\
		Marsabit Z7 & 0.99 & 0.96 & 0.90 \\
		Narok Z15 &  0.99 & 0.94 & 0.83 \\
		Narok Z18 &  0.99 & 0.94 & 0.84 \\
		Samburu Z5 &  0.99 & 0.94 & 0.83 \\
		Samburu Z24 & 0.99 & 0.96 & 0.88 \\
		TanaRiver Z11 &  0.99 & 0.94 & 0.84 \\
		Turkana Z1 &  0.98 & 0.90 & 0.73 \\
		Turkana Z3 &  0.99 & 0.93 & 0.81 \\
		Turkana Z24 & 0.99 & 0.95 & 0.85 \\
		Wajir Z7 & 0.99 & 0.96 & 0.88 \\
		Wajir Z9 &  0.97 & 0.88 & 0.72 \\
		Wajir Z10 &  0.97 & 0.88 & 0.73 \\
		WestPokot Z1 & 0.99 & 0.95 & 0.85 \\
		WestPokot Z24 &  0.99 & 0.95 & 0.85 \\
		\bottomrule
		Median &  0.99 & 0.94 & 0.85 \\
		\bottomrule
	\end{tabular}
\end{table}


% \begin{table}
% 	\footnotesize
% 	\caption{VCI3M forecast performance using AR on the MODIS data. The numbers shown are the percentage standard deviation remaining and the $R^2$ score, respectively. In the 'Forecasts' column, the number gives the percentage of time points for which it was possible to obtain a forecast.  } \label{tab:VCI3m_MODIS}
% 	\centering
% 	%% \tablesize{} %% You can specify the fontsize here, e.g.,  \tablesize{\footnotesize}. If commented out \small will be used.
% 	\begin{tabular}{l|ccc|c} 
% 		\toprule
% 		\textbf{Region}  &  \textbf{2 weeks} &  \textbf{4 weeks}  & \textbf{6 weeks} & \textbf{Forecasts} \\
% 		\midrule
% 		Baringo Z24 & 	6 	0.99 	& 	18 	0.96 	& 	32 	0.89 & 100\\
% 		Elgeyo-Marakwet Z24 &  6 	0.99 	& 	18 	0.96 	& 	32 	0.89 & 95 	\\
% 		Garissa Z10 &  11 	0.98 	& 	20 	0.95 	& 	53 	0.71 & 15	\\
% 		Garissa Z11 & xx xx & xx xx & xx xx & 0 \\
% 		Isiolo Z5 & 12 	0.98 	& 	27 	0.92 	& 	42 	0.81 & 97	 \\
% 		Isiolo Z9 &  13 	0.98 	& 	28 	0.91 	& 	43 	0.81 & 89	\\
% 		Isiolo Z10 & 13 	0.98 	& 	28 	0.91 	& 	42 	0.81 & 71\\
% 		Isiolo Z24 & 8 	0.99 	& 	22 	0.95 	& 	37 	0.86 & 93	\\
% 		Kajiado Z15 & 12 	0.98 	& 	26 	0.92 	& 	39 	0.84 & 75\\
% 		Kajiado Z18& 11 	0.98 	& 	24 	0.94 	& 	38 	0.85 & 71\\
% 		Laikipia Z24 & 9 	0.99 	& 	23 	0.94 	& 	37 	0.85 & 93	\\
% 		Lamu Z11 & xx xx & xx xx & xx xx & 0\\
% 		Mandera Z7 & 15 	0.97 	& 	34 	0.87 	& 	53 	0.71 & 44	\\
% 		Mandera Z9 & 16 	0.97 	& 	35 	0.87 	& 	55 	0.69 & 43	\\
% 		Marsabit Z5 & 9 	0.99 	& 	21 	0.95 	& 	34 	0.88 & 94\\
% 		Marsabit Z7 & 17 	0.96 	& 	35 	0.87 	& 	48 	0.76	& 34		\\
% 		Narok Z15 & 11 	0.98 	& 	28 	0.92 	& 	44 	0.79 & 96\\
% 		Narok Z18& 6 	0.99 	& 	19 	0.96 	& 	32 	0.89 & 98\\
% 		Samburu Z24 & 5 	0.99 	& 	16 	0.97 	& 	30 	0.90 & 100		\\
% 		Samburu Z5 & 7 	0.99 	& 	21 	0.95 	& 	37 	0.86 & 100	\\
% 		Tana River Z11& 11 	0.98 	& 	22 	0.95 	& 	33 	0.88 & 41	\\
% 		Turkana Z1 & 7 	0.99 	& 	20 	0.95 	& 	35 	0.87 & 98 		\\
% 		Turkana Z3 & 7 	0.99 	& 	23 	0.94 	& 	40 	0.83	& 100	\\
% 		Turkana Z24 & 7 	0.99 	& 	20 	0.95 	& 	35 	0.87	& 100	 \\
% 		Wajir Z7 & 14 	0.97 	& 	30 	0.90 	& 	44 	0.79 & 45	\\
% 		Wajir Z9 & 16 	0.97 	& 	33 	0.88 	& 	50 	0.74 & 41	\\
% 		Wajir Z10& 12 	0.98 	& 	23 	0.94 	& 	33 	0.88 & 19	\\
% 		West Pokot Z1 & 7 	0.99 	& 	21 	0.95 	& 	38 	0.85 & 100		\\
% 		West Pokot Z24 & 9 	0.99 	& 	22 	0.94 	& 	36 	0.86 & 94		\\
% 		\bottomrule
% 		Median & 11 0.99 & 23 0.95 & 38 0.85 & 93\\
% 		\bottomrule
% 	\end{tabular}
% \end{table}

\begin{table}
	\footnotesize
	\caption{VCI3M forecast performance ($R^2$-scores) using the AR method on MODIS data. In the `Forecasts' column, the number gives the percentage of time points for which it was possible to obtain a forecast.  } \label{tab:VCI3m_MODIS}
	\centering
	%% \tablesize{} %% You can specify the fontsize here, e.g.,  \tablesize{\footnotesize}. If commented out \small will be used.
	\begin{tabular}{l|ccc|c} 
		\toprule
		\textbf{Region}  &  \textbf{2 weeks} &  \textbf{4 weeks}  & \textbf{6 weeks} & \textbf{Forecasts} \\
		\midrule
		Baringo Z24 &  	0.99 	& 	 	0.96 	&  	0.89 & 100\\
		Elgeyo-Marakwet Z24 &  	0.99 	& 	 	0.96 	&  	0.89 & 95 	\\
		Garissa Z10 &   	0.98 	& 	 	0.95 	&  	0.71 & 15	\\
		Garissa Z11 &  xx &  xx & xx & 0 \\
		Isiolo Z5 &  	0.98 	& 	 	0.92 	&  	0.81 & 97	 \\
		Isiolo Z9 &   	0.98 	& 	 	0.91 	&  	0.81 & 89	\\
		Isiolo Z10 &  	0.98 	& 	 	0.91 	&  	0.81 & 71\\
		Isiolo Z24 & 	0.99 	& 	 	0.95 	&  	0.86 & 93	\\
		Kajiado Z15 &  	0.98 	& 	 	0.92 	&  	0.84 & 75\\
		Kajiado Z18&  	0.98 	& 	 	0.94 	&  	0.85 & 71\\
		Laikipia Z24 & 	0.99 	& 	 	0.94 	&  	0.85 & 93	\\
		Lamu Z11 &  xx & xx & xx & 0\\
		Mandera Z7 & 	0.97 	& 		0.87 	& 		0.71 & 44	\\
		Mandera Z9 &  	0.97 	& 	 	0.87 	&  	0.69 & 43	\\
		Marsabit Z5 & 	0.99 	& 	 	0.95 	&  	0.88 & 94\\
		Marsabit Z7 &  	0.96 	& 	 	0.87 	&  	0.76	& 34		\\
		Narok Z15 &  	0.98 	& 	 	0.92 	&  	0.79 & 96\\
		Narok Z18& 	0.99 	& 	 	0.96 	&  	0.89 & 98\\
		Samburu Z24 & 	0.99 	& 	 	0.97 	&  	0.90 & 100		\\
		Samburu Z5 & 	0.99 	& 	 	0.95 	&  	0.86 & 100	\\
		Tana River Z11&  	0.98 	& 	 	0.95 	&  	0.88 & 41	\\
		Turkana Z1 & 	0.99 	& 	 	0.95 	&  	0.87 & 98 		\\
		Turkana Z3 & 	0.99 	& 	 	0.94 	&  	0.83	& 100	\\
		Turkana Z24 & 	0.99 	& 	 	0.95 	&  	0.87	& 100	 \\
		Wajir Z7 &  	0.97 	& 	 	0.90 	&  	0.79 & 45	\\
		Wajir Z9 &  	0.97 	& 	 	0.88 	&  	0.74 & 41	\\
		Wajir Z10&  	0.98 	& 	 	0.94 	&  	0.88 & 19	\\
		West Pokot Z1 & 	0.99 	& 	 	0.95 	&  	0.85 & 100		\\
		West Pokot Z24 & 	0.99 	& 	 	0.94 	&  	0.86 & 94		\\
		\bottomrule
		Median &  0.99 &  0.95 & 0.85 & 93\\
		\bottomrule
	\end{tabular}
\end{table}


% \begin{table}
% 	\footnotesize
% 	\caption{NDVI anomaly forecast performance using AR on the MODIS data. The numbers shown are the percentage standard deviation remaining and the $R^2$ score, respectively. } \label{tab:NDVI_MODIS}
% 	\centering
% 	%% \tablesize{} %% You can specify the fontsize here, e.g.,  \tablesize{\footnotesize}. If commented out \small will be used.
% 	\begin{tabular}{l|ccc} 
% 		\toprule
% 		\textbf{Region}  &  \textbf{2 weeks} &  \textbf{4 weeks}  & \textbf{6 weeks} \\
% 		\midrule
% 		Baringo Z24 & 32 	0.90 	& 59 	0.65&75 	0.42\\
% 		Elgeyo-Marakwet Z24 &  32 	0.90 &58 	0.66	&73 	0.46 		\\
% 		Garissa Z10 & 42 	0.82 &51 	0.74&56 	0.68	\\
% 		Garissa Z11 & xx xx & xx xx & xx xx  \\
% 		Isiolo Z5 & 55 	0.69&79 	0.37&90 	0.18 		 \\
% 		Isiolo Z9 &  36 	0.87&64 	0.58&80 	0.35 		\\
% 		Isiolo Z10 & 37 	0.86  &65 	0.57&82 	0.32 	\\
% 		Isiolo Z24 & 29 	0.91&57 	0.66&76 	0.42	\\
% 		Kajiado Z15 & 47 	0.78 & 71 	0.48 &82 	0.31 		\\
% 		Kajiado Z18& 45 	0.79&72 	0.48&87 	0.24\\
% 		Laikipia Z24 & 38 	0.85 	&62 	0.60&77 	0.40 		\\
% 		Lamu Z11 & xx xx & xx xx & xx xx \\
% 		Mandera Z7 & 33 	0.89&64 	0.58 &	87 	0.24 		\\
% 		Mandera Z9 & 32 	0.89&65 	0.57 &	90 	0.18 		\\
% 		Marsabit Z5 & 38 	0.85 &64 	0.59&75 	0.44 		\\
% 		Marsabit Z7 & 35 	0.88&	58 	0.66 &71 	0.49 			\\
% 		Narok Z15 & 48 	0.76& 72 	0.48 &83 	0.30 		 \\
% 		Narok Z18&36 	0.87 	& 61 	0.62&72 	0.47\\
% 		Samburu Z24 & 28 	0.92 	&56 	0.68&73 	0.46 		\\
% 		Samburu Z5 & 47 	0.78&74 	0.45 &	86 	0.25 		\\
% 		Tana River Z11& 53 	0.71 &68 	0.54&87 	0.24 		\\
% 		Turkana Z1 &  32 	0.89&	64 	0.58 &	82 	0.32 			\\
% 		Turkana Z3 & 33 	0.89&71 	0.49&88 	0.21 	\\
% 		Turkana Z24 & 31 	0.90 &	62 	0.61 &	80 	0.35 		 \\
% 		Wajir Z7 & 29 	0.91&54 	0.70 &	67 	0.54 		\\
% 		Wajir Z9 & 30 	0.91 &57 	0.67 &	74 	0.44 			\\
% 		Wajir Z10& 26 	0.93 &37 	0.86&46 	0.79 		\\
% 		West Pokot Z1 & 37 	0.86 &	68 	0.53& 85 	0.28	\\
% 		West Pokot Z24 & 42 	0.82&	65 	0.57 &	78 	0.39 		\\
% 		\bottomrule
% 		Median & 36 0.87 & 65 0.58 & 80 0.35 \\
% 		\bottomrule
% 	\end{tabular}
% \end{table}


\begin{table}
	\footnotesize
	\caption{NDVI anomaly forecast performance ($R^2$-scores) using the AR method on MODIS data.} \label{tab:NDVI_MODIS}
	\centering
	%% \tablesize{} %% You can specify the fontsize here, e.g.,  \tablesize{\footnotesize}. If commented out \small will be used.
	\begin{tabular}{l|ccc} 
		\toprule
		\textbf{Region}  &  \textbf{2 weeks} &  \textbf{4 weeks}  & \textbf{6 weeks} \\
		\midrule
		Baringo Z24 & 	0.90 	& 	0.65 & 	0.42\\
		Elgeyo-Marakwet Z24 &  	0.90 &  	0.66	& 	0.46 		\\
		Garissa Z10 & 	0.82 & 	0.74 & 	0.68	\\
		Garissa Z11 & xx &  xx &  xx  \\
		Isiolo Z5 & 	0.69& 	0.37 & 	0.18 		 \\
		Isiolo Z9 &  	0.87&	0.58 &	0.35 		\\
		Isiolo Z10 & 	0.86  & 	0.57 & 	0.32 	\\
		Isiolo Z24 & 	0.91& 	0.66& 	0.42	\\
		Kajiado Z15 & 	0.78 & 	0.48 & 	0.31 		\\
		Kajiado Z18& 	0.79 & 	0.48& 	0.24\\
		Laikipia Z24 & 	0.85 	& 	0.60 & 	0.40 		\\
		Lamu Z11 & xx &  xx &  xx \\
		Mandera Z7 & 	0.89 & 	0.58 &	 	0.24 		\\
		Mandera Z9 & 	0.89 & 	0.57 &	 	0.18 		\\
		Marsabit Z5 & 	0.85 & 	0.59 & 	0.44 		\\
		Marsabit Z7 & 	0.88 & 	0.66 & 	0.49 			\\
		Narok Z15 & 	0.76 & 	0.48 & 	0.30 		 \\
		Narok Z18 &	0.87 	& 	0.62 & 	0.47 \\
		Samburu Z24 & 	0.92 	& 	0.68 & 	0.46 		\\
		Samburu Z5 & 	0.78 & 	0.45 &	 	0.25 		\\
		Tana River Z11& 	0.71 & 	0.54 & 	0.24 		\\
		Turkana Z1 &  	0.89 & 	0.58 &	 	0.32 			\\
		Turkana Z3 & 	0.89& 	0.49 & 	0.21 	\\
		Turkana Z24 & 	0.90 & 	0.61 & 	0.35 		 \\
		Wajir Z7 & 	0.91& 	0.70 &	 	0.54 		\\
		Wajir Z9 & 	0.91 & 	0.67 &	 	0.44 			\\
		Wajir Z10& 	0.93 & 	0.86 &  	0.79 		\\
		West Pokot Z1 & 	0.86 & 	0.53 & 	0.28	\\
		West Pokot Z24 & 	0.82 & 	0.57 & 	0.39 		\\
		\bottomrule
		Median & 0.87 &  0.58 &  0.35 \\
		\bottomrule
	\end{tabular}
\end{table}


% \begin{table}
% 	\footnotesize
% 	\caption{NDVI anomaly forecast performance using GPs on the MODIS data. The numbers shown are the percentage standard deviation remaining and the $R^2$ score, respectively.} \label{tab:NDVI_GPM}
% 	\centering
% 	%% \tablesize{} %% You can specify the fontsize here, e.g.,  \tablesize{\footnotesize}. If commented out \small will be used.
% 	\begin{tabular}{l|ccc} 
% 		\toprule
% 		\textbf{Region}  &  \textbf{2 weeks} &  \textbf{4 weeks}  & \textbf{6 weeks}  \\
% 		\midrule
% 		Baringo Z24 & 37 0.86 & 73 0.44 & 93 0.10 \\
% 		Elgeyo-Marakwet Z24 & 37 0.85 & 73 0.40 & 91 0.07 \\
% 		Garissa Z10 & 46 0.75 & 59 0.50 & 68 0.21 \\
% 		Garissa Z11 & xx xx & xx xx & xx xx \\
% 		Isiolo Z5 & 60 0.60 & 87 0.16 & 96 0.02 \\
% 		Isiolo Z9 & 42 0.83 & 79 0.38 & 97 0.08 \\
% 		Isiolo Z10 & 41 0.82 & 77 0.36 & 95 0.05 \\
% 		Isiolo Z24 & 34 0.89 & 70 0.51 & 92 0.15 \\
% 		Kajiado Z15 & 48 0.75 & 76 0.35 & 90 0.09 \\
% 		Kajiado Z18 & 47 0.78 & 77 0.40 & 94 0.11 \\
% 		Laikipia Z24 & 43 0.81 & 72 0.46 & 89 0.16 \\
% 		Lamu Z11 & xx xx & xx xx & xx xx\\
% 		Mandera Z7 & 39 0.86 & 74 0.50 & 97 0.17 \\
% 		Mandera Z9 & 35 0.89 & 71 0.54 & 96 0.20 \\
% 		Marsabit Z5 & 42 0.79 & 69 0.37 & 80 0.11 \\
% 		Marsabit Z7 & 37 0.80 & 61 0.38 & 74 0.09 \\
% 		Narok Z15 & 50 0.72 & 76 0.32 & 87 0.09 \\
% 		Narok Z18 & 39 0.84 & 74 0.42 & 90 0.11 \\
% 		Samburu Z5 & 50 0.70 & 80 0.26 & 91 0.05 \\
% 		Samburu Z24 & 34 0.88 & 71 0.48 & 92 0.12 \\
% 		TanaRiver Z11 & 60 0.47 & 75 0.13 & 87 -0.00 \\
% 		Turkana Z1 & 36 0.87 & 75 0.44 & 95 0.10 \\
% 		Turkana Z3 & 35 0.88 & 80 0.37 & 99 0.03 \\
% 		Turkana Z24 & 35 0.87 & 75 0.42 & 95 0.07 \\
% 		Wajir Z7 & 31 0.86 & 57 0.45 & 72 0.15 \\
% 		Wajir Z9 & 36 0.85 & 65 0.45 & 79 0.15 \\
% 		Wajir Z10 & 37 0.77 & 53 0.37 & 68 -0.17 \\
% 		WestPokot Z1 & 41 0.83 & 77 0.39 & 96 0.08 \\
% 		WestPokot Z24 & 49 0.78 & 86 0.34 & 103 0.06 \\
% 		\bottomrule
% 		Median & 39 0.83 & 74 0.40 & 92 0.09 \\
% 		\bottomrule
% 	\end{tabular}
% \end{table}

\begin{table}
	\footnotesize
	\caption{NDVI anomaly forecast performance ($R^2$-scores) using the GP method on MODIS data.} \label{tab:NDVI_GPM}
	\centering
	%% \tablesize{} %% You can specify the fontsize here, e.g.,  \tablesize{\footnotesize}. If commented out \small will be used.
	\begin{tabular}{l|ccc} 
		\toprule
		\textbf{Region}  &  \textbf{2 weeks} &  \textbf{4 weeks}  & \textbf{6 weeks}  \\
		\midrule
		Baringo Z24 &  0.86 &  0.44 & 0.10 \\
		Elgeyo-Marakwet Z24 &  0.85 &  0.40 & 0.07 \\
		Garissa Z10 &  0.75 &  0.50 & 0.21 \\
		Garissa Z11 & xx & xx &  xx \\
		Isiolo Z5 &  0.60 &  0.16 & 0.02 \\
		Isiolo Z9 &  0.83 &  0.38 & 0.08 \\
		Isiolo Z10 &  0.82 &  0.36 & 0.05 \\
		Isiolo Z24 &  0.89 &  0.51 & 0.15 \\
		Kajiado Z15 &  0.75 &  0.35 & 0.09 \\
		Kajiado Z18 &  0.78 &  0.40 & 0.11 \\
		Laikipia Z24 &  0.81 &  0.46 & 0.16 \\
		Lamu Z11 &  xx &  xx &  xx\\
		Mandera Z7 &  0.86 &  0.50 & 0.17 \\
		Mandera Z9 &  0.89 &  0.54 & 0.20 \\
		Marsabit Z5 &  0.79 &  0.37 & 0.11 \\
		Marsabit Z7 &  0.80 &  0.38 & 0.09 \\
		Narok Z15 &  0.72 &  0.32 & 0.09 \\
		Narok Z18 &  0.84 &  0.42 & 0.11 \\
		Samburu Z5 &  0.70 &  0.26 & 0.05 \\
		Samburu Z24 &  0.88 &  0.48 & 0.12 \\
		TanaRiver Z11 &  0.47 &  0.13 & -0.00 \\
		Turkana Z1 &  0.87 &  0.44 & 0.10 \\
		Turkana Z3 &  0.88 &  0.37 & 0.03 \\
		Turkana Z24 &  0.87 &  0.42 & 0.07 \\
		Wajir Z7 &  0.86 &  0.45 & 0.15 \\
		Wajir Z9 &  0.85 &  0.45 & 0.15 \\
		Wajir Z10 &  0.77 &  0.37 & -0.17 \\
		WestPokot Z1 &  0.83 &  0.39 & 0.08 \\
		WestPokot Z24 &  0.78 &  0.34 &3 0.06 \\
		\bottomrule
		Median &  0.83 &  0.40 & 0.09 \\
		\bottomrule
	\end{tabular}
\end{table}

% \begin{table}
% 	\footnotesize
% 	\caption{NDVI anomaly forecast performance using AR on Landsat data. The numbers shown are the percentage standard deviation remaining and the $R^2$ score, respectively.} \label{tab:NDVI_Landsat_AR}
% 	\centering
% 	%% \tablesize{} %% You can specify the fontsize here, e.g.,  \tablesize{\footnotesize}. If commented out \small will be used.
% 	\begin{tabular}{l|ccc} 
% 		\toprule
% 		\textbf{Region}  &  \textbf{2 weeks} &  \textbf{4 weeks}  & \textbf{6 weeks}  \\
% 		\midrule
% 		Baringo Z24 	& 	67 	0.54 	& 	89 	0.19 	& 	105 	-0.12\\
% 		Elgeyo-Marakwet Z24 	& 	74 	0.44 	& 	98 	0.03 	& 	113 	-0.28\\
% 		Garissa Z10 	& 	77 	0.39 	& 	96 	0.07 	& 	108 	-0.17\\
% 		Garissa Z11 	& 	78 	0.38 	& 	94 	0.10 	& 	106 	-0.13\\
% 		Isiolo Z5 	& 	86 	0.25 	& 	103 	-0.06 	& 	114 	-0.31\\
% 		Isiolo Z9 	& 	96 	0.06 	& 	109 	-0.20 	& 	116 	-0.35\\
% 		Isiolo Z10 	& 	108 	-0.17 	& 	120 	-0.46 	& 	126 	-0.60\\
% 		Isiolo Z24 	& 	66 	0.55 	& 	82 	0.31 	& 	92 	0.13\\
% 		Kajiado Z15 	& 	59 	0.64 	& 	78 	0.38 	& 	91 	0.15\\
% 		Kajiado Z18 	& 	60 	0.63 	& 	76 	0.40 	& 	90 	0.18\\
% 		Laikipia Z24 	& 	54 	0.70 	& 	75 	0.42 	& 	93 	0.12\\
% 		Lamu Z11 	& 	88 	0.20 	& 	97 	0.05 	& 	103 	-0.07\\
% 		Mandera Z7 	& 	78 	0.38 	& 	87 	0.23 	& 	93 	0.12\\
% 		Mandera Z9 	& 	56 	0.67 	& 	69 	0.51 	& 	76 	0.41\\
% 		Marsabit Z5 	& 	76 	0.41 	& 	91 	0.15 	& 	103 	-0.07\\
% 		Marsabit Z7 	& 	52 	0.72 	& 	66 	0.55 	& 	78 	0.38\\
% 		Narok Z15 	& 	79 	0.36 	& 	101 	-0.02 	& 	109 	-0.19\\
% 		Narok Z18 	& 	73 	0.46 	& 	91 	0.15 	& 	101 	-0.02\\
% 		Samburu Z5 	& 	74 	0.44 	& 	95 	0.08 	& 	110 	-0.21\\
% 		Samburu Z24 	& 	60 	0.63 	& 	81 	0.32 	& 	97 	0.05\\
% 		TanaRiver Z11 	& 	90 	0.17 	& 	108 	-0.18 	& 	120 	-0.46\\
% 		Turkana Z1 	& 	80 	0.35 	& 	103 	-0.07 	& 	118 	-0.39\\
% 		Turkana Z3 	& 	75 	0.43 	& 	95 	0.09 	& 	111 	-0.24\\
% 		Turkana Z24 	& 	69 	0.52 	& 	91 	0.16 	& 	106 	-0.14\\
% 		Wajir Z7 	& 	53 	0.70 	& 	63 	0.59 	& 	70 	0.50\\
% 		Wajir Z9 	& 	77 	0.40 	& 	85 	0.27 	& 	84 	0.27\\
% 		Wajir Z10 	& 	100 	0.00 	& 	111 	-0.24 	& 	117 	-0.37\\
% 		WestPokot Z1 	& 	70 	0.50 	& 	92 	0.15 	& 	107 	-0.15\\
% 		WestPokot Z24 	& 	73 	0.45 	& 	96 	0.06 	& 	114 	-0.31\\
% 		\bottomrule
% 		Median & 74 0.44 & 92 0.15 & 106 -0.13 \\
% 		\bottomrule
% 	\end{tabular}
% \end{table}


\begin{table}
	\footnotesize
	\caption{NDVI anomaly forecast performance ($R^2$-scores) using the AR method on Landsat data.} \label{tab:NDVI_Landsat_AR}
	\centering
	%% \tablesize{} %% You can specify the fontsize here, e.g.,  \tablesize{\footnotesize}. If commented out \small will be used.
	\begin{tabular}{l|ccc} 
		\toprule
		\textbf{Region}  &  \textbf{2 weeks} &  \textbf{4 weeks}  & \textbf{6 weeks}  \\
		\midrule
		Baringo Z24 	& 	0.54 	& 	0.19 	& 		-0.12\\
		Elgeyo-Marakwet Z24 	& 	0.44 	& 	0.03 	& 		-0.28\\
		Garissa Z10 	& 	0.39 	& 	0.07 	& 		-0.17\\
		Garissa Z11 	& 	0.38 	& 	0.10 	& 		-0.13\\
		Isiolo Z5 	& 	0.25 	&  	-0.06 	 & 	-0.31\\
		Isiolo Z9 	& 	0.06 	&  	-0.20 	& 	-0.35\\
		Isiolo Z10 	& 		-0.17 	&  	-0.46 	& 	-0.60\\
		Isiolo Z24 	& 	0.55 	& 	0.31 	& 	0.13\\
		Kajiado Z15 	& 	0.64 	& 	0.38 	& 	0.15\\
		Kajiado Z18 	& 	0.63 	& 	0.40 	& 	0.18\\
		Laikipia Z24 	& 	0.70 	& 	0.42 	& 	0.12\\
		Lamu Z11 	& 	0.20 	& 	0.05 	& 		-0.07\\
		Mandera Z7 	& 	0.38 	& 	0.23 	& 	0.12\\
		Mandera Z9 	& 	0.67 	& 	0.51 	& 	0.41\\
		Marsabit Z5 	& 	0.41 	& 	0.15 	& 		-0.07\\
		Marsabit Z7 	& 	0.72 	& 	0.55 	& 	0.38\\
		Narok Z15 	& 	0.36 	&  	-0.02 	& -0.19\\
		Narok Z18 	& 	0.46 	& 	0.15 	& 	-0.02\\
		Samburu Z5 	& 	0.44 	& 	0.08 	& 	-0.21\\
		Samburu Z24 	& 	0.63 	& 	0.32 	& 	0.05\\
		TanaRiver Z11 	& 	0.17 	&  	-0.18 	& 	-0.46\\
		Turkana Z1 	& 	0.35 	&  	-0.07 	& 	-0.39\\
		Turkana Z3 	& 	0.43 	& 	0.09 	& 		-0.24\\
		Turkana Z24 	& 	 	0.52 	& 	0.16 	&  	-0.14\\
		Wajir Z7 	& 	0.70 	& 	0.59 	& 	0.50\\
		Wajir Z9 	& 	0.40 	& 	0.27 	& 	0.27\\
		Wajir Z10 	&  	0.00 	&  	-0.24 	& 	-0.37\\
		WestPokot Z1 	& 	0.50 	& 	0.15 	& 		-0.15\\
		WestPokot Z24 	& 	0.45 	& 	0.06 	& 		-0.31\\
		\bottomrule
		Median & 0.44 & 0.15 &  -0.13\\
		\bottomrule
	\end{tabular}
\end{table}

\newpage

\subsection{Assessing forecasting with the inclusion of additional variables} 

For the MODIS data, we tested to see whether we could improve the prediction of NDVI anomaly by including the past of other available variables in the AR model, i.e.~we performed a Granger causality analysis. Taking $X$ as NDVI anomaly, as in equation \eqref{eq:AR1}, for another variable $Y$, the extended model was fit:
\begin{equation}
X_{t+n}=\sum_{i=0}^{p-1}a_iX_{t-i}+\sum_{i=0}^{q-1}b_iY_{t-i}+\epsilon^{\prime}_t\,,
\end{equation}
and Granger causality measured as $\Delta R^2$, the $R^2$-score obtained from this extended model minus the $R^2$-score obtained  from the previous (reduced) model \eqref{eq:AR1}. 

Firstly, we tested whether including past observations of either the red band or the NIR band (at the same lags as NDVI anomaly) in the regression to predict NDVI anomaly could improve the quality of the forecast, and found it did not. For a lead time of 4 weeks, for example, the improvement in $R^2$-score was generally negative; the mean improvement across regions was -0.007 for red and -0.01 for NIR.

Secondly, we tested for Granger causality of NDVI anomaly from each region to each other region (within the set of regions for which predictions could be made more than 50\% of the time). That is, for each pair of distinct regions, $i$ and $j$, the 3 most recent observations from region $j$ were added to the AR forecast model for region $i$, and the $R^2$-score was compared with that obtained without including observations from region $j$. There was not strong Granger causality of NDVI anomaly between most regions. For only a few combinations was there an improvement in $R^2$-score of more than 0.05, see Fig.~\ref{fig:GCheatmap}. Nevertheless, these results suggest that, to create the optimal linear regression based forecasting method, data from all regions should be used. Future work will explore how best to extract any useful information from regions other than the one being forecast.



\begin{figure} 
	\centering
	\includegraphics[trim = 20mm 0mm 0mm 0mm,width=11 cm]{figures/GCheatmap2.pdf}
	\caption{Granger causality of NDVI anomaly from each region to each other region, computed on the MODIS data, measured as improvement in $R^2$-score when observations from region `From' are added to the AR model for forecasting region `To' at a lead time of 4 weeks. Only substantial Granger causalities are shown, i.e.~those with $\Delta R^2>0.05$.} \label{fig:GCheatmap}
\end{figure}



% \newpage
% \section*{References}

% \bibliography{mybibfile}

\newpage
\listoffigures

\listoftables

% \newpage
% \section*{Page identifications for responses to reviewers }

% \begin{itemize}
%     \item Sketch illustrating the Gaussian Process, Line~\ref{review:GP_sketchL}, Page~\pageref{review:GP_sketch}, Figure~\ref{fig:GP_illustration}
%     \item Thoroughly revised Discussion Section~\ref{sec:dis}, Page~\pageref{sec:dis}, Line~\ref{review:dis}
%     %
%         \item New Caveats and Future work section Section~\ref{sec:futurework}, Page~\pageref{sec:futurework}, Line~\ref{review:futurework}
%         \item Explanation of the why the method works due to temporal correlations.  Page~\pageref{review:correlation}, Line~\ref{review:correlationL}

% \end{itemize}


\end{document}








%\paragraph{Usage} Once the package is properly installed, you can use the document class \emph{elsarticle} to create a manuscript. Please make sure that your manuscript follows the guidelines in the Guide for Authors of the relevant journal. It is not necessary to typeset your manuscript in exactly the same way as an article, unless you are submitting to a camera-ready copy (CRC) journal.

%The author names and affiliations could be formatted in two ways:
%\begin{enumerate}[(1)]
%\item Group the authors per affiliation.
%\item Use footnotes to indicate the affiliations.
%\end{enumerate}


